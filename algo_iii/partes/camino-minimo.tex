\section{Caminos mínimos}
Los grafos (pesados o no) son la estructura natural para modelar redes en las cuales uno quiere ir de un punto a otro de la red atravesando una secuencia de enlaces. Formalmente, sea \(G = (V, X)\) un grafo y \(l: X \to \reales\) una función de longitud/peso para las aristas de \(G\):

\begin{itemize}
  \item La \textbf{longitud} de un camino \(C\) entre dos vértices \(v\) y \(w\) es la suma de las longitudes de las aristas del camino:
        \[l(C) = \sum_{e\in C}l(e)\]
  \item Un \textbf{camino mínimo} \(C^0\) entre \(v\) y \(w\) es un caminot entre \(v\) y \(w\) tal que \[l(C^0)=\min\{l(C) | C \text{es un camino entre } v \text{ y } w\}\]
\end{itemize}

Dado un grafo \(G\), se pueden definir tres variantes de problemas sobre caminos mínimos:\
\begin{itemize}
  \item \textbf{Único origen - único destino:} Determinar un camino mínimo entre dos vértices específicos \(v\) y \(w\).
  \item \textbf{Único origen - múltiples destinos:} Determinar un camino mínimo desde un vértice específico \(v\) al resto de los vértices de \(G\).
  \item \textbf{Múltiples origenes - múltiples destinos:} Determinar una camino mínimo entre todos los vértices de \(G\).
\end{itemize}

Generalmente, los algoritmos para resolver problemas de camino mínimo se basan en que todo subcamino de un camino mínimo entre dos vértices es un camino mínimo.

\begin{proposicion}
  Dado un grafo \(G = (V, X)\) con una función de peso \(l: X \to \reales\), sea \(P: v_1\dots v_k\) un camino mínimo de \(v_1\) a \(v_k\). Entonces \(\forall~1\leq i \leq j \leq k\), \(P_{v_iv_j}\) es un camino mínimo desde \(v_i\) a \(v_j\).
\end{proposicion}
\begin{demo}
  Podemos descomponer al camino \(P\) en \(P_{v_1v_i} + P_{v_iv_j} + P_{v_jv_k}\), entonces \(l(P) = l(P_{v_1v_i}) + l(P_{v_iv_j})) + l(P_{v_jv_k})\).

  Por el absurdo, asumamos que \(P_{v_iv_j}\) no es un camino mínimo desde \(v_i\) a \(v_j\). Llamemos \(P'\) a un camino entre estos dos vértices, entonces \(l(P') \leq P_{v_iv_j}\). Entonces podemos armar \(P'' = P_{v_1v_i} + P' + P_{v_jv_k}\) y
  \[l(P'') = l(P_{v_1v_i}) + l(P') + l(P_{v_jv_k}) < l(P_{v_1v_i}) + l(P_{v_iv_j})) + l(P_{v_jv_k}) = l(P)\]

  Luego \(P\) no es un camino mínimo entre \(v\) y \(w\).
\end{demo}

Dos consideraciones a tener en cuenta:
\begin{itemize}
  \item \textbf{Aristas con peso negativo:} Si el digrafo \(G\) no contiene ciclos de peso negativo o contiene alguno pero no es alcanzable desde \(v\), entonces el problema sigue estando bien definido, aunque algunos caminos puedan tener longitud negativa. Sin embargo, si \(G\) tiene algún circuito de peso negativo alcanzable desde \(v\), el concepto de camino mínimo deja de estar bien definido: Si \(w\) pertenece a un camino con ciclo de peso negativo, ningún camino de \(v\) a \(w\) puede ser mínimo porque siempre se puede obtener un camino de peso menor siguiendo ese camino pero atravesando el ciclo de peso negativo una vez más.
  \item \textbf{Circuitos:} Siempre existe un camino mínimo que no contiene circuitos (si el problema está bien definido):
        \begin{itemize}
          \item Sabemos que no puede tener un circuito negativo porque si no no está bien definido.
          \item Si tiene un circuito de peso positivo, entonces podemos sacarlo obteniendo un camino con el mismo origen y destino de menor longitud (osea que no era camino mínimo).
          \item Si tiene un circuito de peso cero, entonces sacandolo obtenemos un camino sin cirtutos del mismo peso.
        \end{itemize}
\end{itemize}

\subsection{Camino mínimo con un único origen y múltiples destinos}
Dado \(G = (V, X)\) un digrafo, \(l: X\to\reales\) una función que asigna a cada arco una cierta longitud y \(v\in V\) un vértice del digrafo, queremos calcular los caminos mínimos desde \(v\) al resto de los vértices.

\subsubsection{En digramos no pesados}
En el caso de que todas las aristas tengan igual longitud, este problema se traduce en encontrar los caminos que definan las distancias (de menor cantidad de aristas). Para esto se puede adaptar el BFS (Sección \ref{algorithm:BFS}):

\begin{algorithmic}
  \Procedure{BFS}{$\mathtt{G = (V, X)}$, $v \in V$}
  \State $\mathtt{pred[w] =}$ \texttt{antecesor de w en un camino minimo desde v}
  \State $\mathtt{dist[w] =}$ \texttt{distancia desde v a w}

  \State $\mathtt{pred[v]\gets 0}$
  \State $\mathtt{dist[v]\gets 0}$
  \State $\mathtt{to\_visit \gets \{v\}}$
  \For{$\mathtt{\forall~w\in V \backslash \{ v }$}
  \State $\mathtt{dist[w] \gets \infty}$
  \EndFor
  \While{$\mathtt{\lnot to\_visit.empty()}$}
  \State $\mathtt{x \gets to\_visit.pop()}$
  \For{$\mathtt{\forall~w \texttt{ tal que } (x \to w)\in X~/~\land dist[w] = \infty}$}
  \State $\mathtt{pred[w] \gets x}$
  \State $\mathtt{dist[w] \gets dist[x] + 1}$
  \State $\mathtt{to\_visite.push(w)}$
  \EndFor
  \EndWhile
  \State\Return \texttt{[pred, dist]}
  \EndProcedure
\end{algorithmic}