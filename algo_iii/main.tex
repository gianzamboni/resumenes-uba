\usepackage{amsmath}

\usepackage{inputenc}
\usepackage{fvextra}

\usepackage{algpseudocode}
\usepackage{algorithmicx}
\usepackage{amsfonts}
\usepackage{amssymb}
\usepackage{amsthm}
\usepackage[spanish]{babel}
\usepackage{bm}
\usepackage{booktabs} % To thicken table lines
\usepackage{bussproofs}
\usepackage{caption}
\usepackage{csquotes}
\usepackage{colortbl}
\usepackage{dsfont}
\usepackage{environ}
\usepackage[shortlabels]{enumitem}
\usepackage{fancyhdr}
\usepackage{forest}
\usepackage{geometry}
\usepackage{graphicx}
\usepackage[hidelinks]{hyperref}
\usepackage{ifthen}
\usepackage{multicol}
\usepackage{multirow}
\usepackage{sidecap}
\usepackage{stmaryrd}
\usepackage{tabularx}
\usepackage{titling}
\usepackage{tikz}
\usepackage{xcolor}
\usepackage{wrapfig}
\usepackage{minted}

\usetikzlibrary{arrows}
\usetikzlibrary{arrows.meta}
\usetikzlibrary{automata}
\usetikzlibrary{calc}
\usetikzlibrary{fit}
\usetikzlibrary{matrix}
\usetikzlibrary{positioning}
\usetikzlibrary{shapes.geometric}
\usetikzlibrary{shapes.multipart}

\usemintedstyle{borland}

\newcommand{\red}[1]{{\color{red}#1}}
\newcommand{\green}[1]{{\color{green}#1}}
\newcommand{\blue}[1]{{\color{blue}#1}}
\newcommand{\violet}[1]{{\color{violet}#1}}
\newcommand{\orange}[1]{{\color{orange}#1}}

\newcommand{\nat}{\mathbb{N}}
\newcommand{\reales}{\mathbb{R}}

\newtheorem{theorem}{Teorema}
\newtheorem{coro}{Corolario}
\newtheorem{proposicion}{Proposición}
\newtheorem{lema}{Lema}


\usetikzlibrary{shapes.multipart}

\tikzstyle{demoBox} = [
draw=blue!20, very thick,
rectangle split, rectangle split parts=2, rounded corners, inner xsep=0.5cm,
rectangle split part fill = {blue!20, blue!5}
]

\tikzstyle{demoPart} = [
draw=blue!20, very thick,
rounded corners, inner xsep=0.5cm,
fill = blue!5
]
%\newcommand{\qed}{\begin{flushright}
%		$\blacksquare$
%\end{flushright}}

\NewEnviron{demo}[1][]{%
	\begin{center}
		\begin{tikzpicture}
			\node [demoBox](box){%
				\textbf{\scriptsize
					DEMOSTRACIÓN #1}
				\nodepart{two}
				\begin{minipage}{0.75\textwidth}
					\vspace*{0.1cm}
					\BODY
				\end{minipage}
			};
		\end{tikzpicture}
	\end{center}
}

\NewEnviron{demoPart}[1][]{%
	\begin{center}
		\begin{tikzpicture}
			\node [demoPart](box){%
				\begin{minipage}{0.75\textwidth}
					\vspace*{0.1cm}
					\BODY
				\end{minipage}
			};
		\end{tikzpicture}
	\end{center}
}


\title{Algoritmos III - Apuntes para final}
\author{Gianfranco Zamboni}

\usepackage[backend=biber,style=chem-acs,sorting=none]{biblatex}
\nocite{*}

\addbibresource{bibliography.bib}

\usemintedstyle[cpp]{bordeland, tabsize=2}
%%%% CONFIGURACIONES %%%%

%% La coma de los reales es un punto
\decimalpoint{}

%%% Tamaño de pagina
%\geometry{
%	includeheadfoot,
%	left=2.54cm,
%	bottom=1cm,
%	top=1cm,
%	right=2.54cm
%}

%\stul{0.1cm}{0.2ex}

%% HEADER Y FOOTER
\pagestyle{fancy}

\fancyhf{}

\fancyhead[LO]{\rightmark} % \thesection\ 
\fancyhead[RO]{\small{\thetitle}}
\fancyfoot[CO]{\thepage}
\renewcommand{\headrulewidth}{0.5pt}
\renewcommand{\footrulewidth}{0.5pt}
\setlength{\headsep}{1cm}
\setlength{\headheight}{13.07225pt}

\renewcommand{\baselinestretch}{1.2}  % line spacing

%% Links en indice 
\hypersetup{
	linktoc=all,     %set to all if you want both sections and subsections linked
	linkcolor=blue,  %choose some color if you want links to stand out
}
\setcounter{tocdepth}{3}

\begin{document}
	
  
\usetikzlibrary{positioning}
\usetikzlibrary{shapes.geometric}
\usetikzlibrary{arrows}
\usetikzlibrary{arrows.meta}
\usetikzlibrary{fit}
\usetikzlibrary{calc}
\usetikzlibrary{matrix}
\usetikzlibrary{graphs, graphs.standard, quotes}
\usetikzlibrary{babel}

\tikzstyle{graphStyle}=[
	node distance=0.75cm
]

\tikzstyle{emptyDemoNode} = [
draw=none, very thick,
rectangle split, rectangle split parts=1, rounded corners, inner xsep=0cm, inner ysep=0cm,
rectangle split part fill = {blue!5}
]


\tikzstyle{basicNode}=[
	circle, 
	draw=black,
	thick,
	fill=green!50,
]

\tikzstyle{redNode}=[
basicNode,
fill=red!70,	
]

\tikzstyle{redEdge}=[
red!70,	
line width=0.75mm,
]

\tikzstyle{blueNode}=[
basicNode,
fill=blue!70,	
]

\tikzstyle{blueEdge}=[
blue!70,	
line width=0.75mm,
]

\tikzstyle{violetNode}=[
basicNode,
fill=violet!70,	
]

\tikzstyle{violetEdge}=[
violet!70,	
line width=0.75mm,
]

\tikzstyle{orangeNode}=[
basicNode,
fill=orange!70,	
]

\tikzstyle{orangeEdge}=[
orange!70,	
line width=0.75mm,
]
\tikzstyle{smallNode}=[
circle, 
draw=black,
thick,
fill=green!50,
minimum size=0.5,
scale=0.75
]


	
	\maketitle
	\tableofcontents

\newpage
\section{Introducción}
\subsection{Relaciones}
Dados dos conjuntos \(A\) y \(B\), se llama \textbf{relación} \(R: A \to B\) de \(A\) en \(B\) a todo subconjutno de \(A\times B\), es decir \(R\subset A\times B\).

Dos elementos \(a\in A\) y \(b\in B\) están relacionados si \((a,b)\in R\) y lo notamos \(aRb\).

Si \(A = B\), se dice que \(R\) es una relación sobre \(A\) y se dice que:
\begin{itemize}
  \item es \textbf{reflexica} cuando \(\forall a,~aRa\).
  \item es \textbf{simétrica} cuando \(\forall a,b\in A,~aRb \implies bRa\).
  \item es \textbf{transitiva} cuando \(a,b,c\in A,~aRb~\land~bRc\implies aRc\).
\end{itemize}

\paragraph{Relación de equivalencia:} Una relación \(R:A\to A\) es de \textbf{equivalencia} cuando es reflexiva, simétrica y transitiva. Este tipo de relaciones particiona a \(A\) en subconjuntos disjuntos llamados \textbf{clases de equivalencia}.


\subsubsection{Operaciones}
\paragraph{Composición de relaciones:} Si \(R:A\to B\) y \(S:B\to C\) son relaciones, entonces la composición de \(R\) y \(S\) es la relación \(S\circ R:A\to C\) definida por:
\[S\circ R = \{(a,c)~|~a\in A,~c\in C : \exists b\in B,~aRb~\land~bSc\}\]

\paragraph{Relación de identidad:} La relación de identidad sobre \(A\) es la relación \(id_A:A\to A\) definida por: \(id_A = \{(a,a)~|~a\in A\}\).
\begin{itemize}
  \item La relación de identidad el el elemento neutro de la composición de relaciones.
\end{itemize}

\paragraph{Relación de potencia:} Dado \(R: A\to A\) se define la relación de potencia \(R^k: A\to A\) como la composición de \(k\) copias de \(R\):
\[R^n = \left\{
  \begin{array}{ll}
    id_A           & \text{si } n = 0 \\
    R\circ R^{n-1} & \text{si } n > 0
  \end{array}
  \right.
\]

\paragraph{Clausura transitiva/positiva:} Dada una relación \(R:A\to A\) se define la clausura transitiva de \(R\) como la relación \(R^+\) definida por: \[R^+ = \bigcup_{n=1}^\infty R^n\]

La clausura transitiva de \(R\) cumple las siguientes propiedades:
\begin{enumerate}
  \item \(R\subseteq R^+\)
        \newpage
  \item \(R^+\) es transitiva
        \begin{demo}[0.86\textwidth]
          ~Si \(a R^+ b\) entonces existe una secuencia de elementos \(a = \red{a_0, a_1, \dots, a_n} = b\) tales que \(\red{a_i} R \red{a_{i+1}}\) para todo \(i\in [0,n-1]\).

          \vspace*{0.25cm}
          Análogamente, como \(b R^+ c\) existe una secuencia de elementos \(b = \blue{b_0, b_1, \dots, b_m} = c\) tales que \(\blue{b_i} R \blue{b_{i+1}}\) para todo \(i\in [0,m-1]\).

          \vspace*{0.25cm}
          Entonces \(a R^{n+m} c\) pues puedo armar la secuencia \(a = \red{a_0, a_1, \dots, a_n},\blue{b_1\dots b_m} = c\).

          \vspace*{0.25cm}
          Luego como \(R^{n+m}\subseteq R^+\) vale que \(a R^+ c\).
        \end{demo}

  \item Para toda relación \(G:A\to A\) tal que \(R\subseteq G \land G\) es transitiva, entonces \(R^+\subseteq G\), es decir \(R^+\) es la relación transitiva más pequeña que contiene a \(R\).
        \begin{demo}[0.86\textwidth]
          Si \(a R^+ b\) entonces existe una secuencia de elementos \(a = a_0, a_1, \dots, a_n = b\) tales que \(a_i R a_{i+1}\) para todo \(i\in [0,n-1]\).

          \vspace*{0.25cm}
          Como \(R\subseteq G\) entonces \(a_i G a_{i+1}\) para todo \(i\in [0,n-1]\). Como \(G\) es transitiva entonces la aplicación repetida de la transitividad nos lleva a que \(a_1 G a_n\), por lo que \(a G b\).
        \end{demo}
\end{enumerate}

\paragraph{Clausura transitiva reflexiva:} \[ R^* = R^+ \cup id_A = \bigcup_{n=0}^\infty R^n\]

\paragraph{Observaciones:}
\begin{itemize}
  \item Si \(A\) es un conjunto finito, entonces todas las relaciones \(R:A\to A\) son finitas.
  \item Si \(R\) es reflexiva, entonces \(R^* = R^+\).
\end{itemize}



\subsection{Alfabetos}
\paragraph{Alfabeto:} Un alfabeto es un conjunto finito de símbolos.

\paragraph{Cadena:} Una cadena sobre un alfabeto \(\Sigma\) es una secuencia finita de símbolos de \(\Sigma\). Los símbolos son notados respetando el orden de la secuencia.

\paragraph{Concatenación:} Es una operación entre un símbolo del alfabeto \(\Sigma\) y una cadena sobre dicho alfabeto:
\[ \circ : \Sigma\times\{\text{cadenas sobre }\Sigma\}\to\{\text{cadenas de }\Sigma\}\]
\begin{itemize}
  \item La cadena nula \(\lambda\) es el elemento neutro de la concatenación.
\end{itemize}

\paragraph{Clausura de Kleene de \(\Sigma\): \(\Sigma^*\)}
\begin{itemize}
  \item \(\lambda\in\Sigma^*\)
  \item \(\alpha\in\Sigma^*\implies \forall~a\in\Sigma,~a\circ\alpha\in\Sigma^*\)
\end{itemize}

\paragraph{Clausura positiva de \(\Sigma\):} \(\Sigma^+ = \Sigma^*\setminus\{\lambda\}\)

\subsection{Lenguajes}
\paragraph{Lenguaje:} Un lenguaje es un conjunto de cadenas sobre un alfabeto \(\Sigma\).

\paragraph{Concatenación de lenguajes:} Si \(L_1\) y \(L_2\) son lenguajes definidos sobre los alfabetos \(\Sigma_1\) y \(\Sigma_2\) respectivamente, entonces la concatenación de \(L_1\) y \(L_2\) es un lenguaje \(L_1L_2\) sobre el alfabeto \( \Sigma_1\cup\Sigma_2\) dfinido de la siguiente manera:
\[ L_1L_2 = \{ \alpha\beta~|~\alpha\in L_1,~\beta\in L_2\}\]

\paragraph*{Clausura de Kleene \(L^*\):}
\begin{itemize}
  \item[] \(L^0 = \{\lambda\}\)
  \item[] \(L^n = LL^{n-1}\) para \(n>=1\)
  \item[] \(L^* = \overset{\infty}{\underset{n=0}{\bigcup}} L^n\)
\end{itemize}

\paragraph{Clausura positiva \(L^+\):}
\begin{itemize}
  \item[] \(L^+ = \overset{\infty}{\underset{n=1}{\bigcup}} L^n\)
\end{itemize}
\paragraph*{Observaciones:}
\begin{itemize}
  \item \(L^+ = LL^*\)
  \item \(L^* = L^+\cup\{\lambda\}\)
  \item Si \(L\) es un lenguaje definido sobre \(\Sigma\) entonces \(L\subseteq\Sigma^*\)
\end{itemize}

\subsection{Gramáticas}
Una gramática es una 4-tupla \((V_N,V_T,P,S)\) donde:
\begin{itemize}
  \item \(V_N\) es un conjunto finito de símbolos no terminales.
  \item \(V_T\) es un conjunto finito de símbolos terminales.
  \item \(P\) es un conjunto finito de reglas de producción: Son pares ordenados \(\alpha\to \beta\) donde \[\alpha\in(V_N\cup V_T)^*V_N(V_N\cup V_T)^*\text{ y }\beta\in(V_N\cup V_T)^*\]
  \item \(S\in V_N\) es el símbolo inicial.
\end{itemize}

Dada una producción \(A\to\alpha\in P\), se denomina a \(A\) como \textbf{cabeza} de la producción y a \(\alpha\) como su \textbf{cuerpo}.

\paragraph{Derivación:} El el proceso por el cual se obtiene una cadena a partir de un símbolo inicial remplazando recursivamente símbolos no terminales por cuerpos de producciones en \(P\) cuya cabeza coincida con los símbolos que están siendo remplazados.

\paragraph{Forma setencial de una grámatica:} Se llama forma sentencial a cualquier derivación de la grámatica:
\begin{itemize}
  \item \(S\) es una forma setencial de \(G\)
  \item Si \(\alpha\beta\gamma\) es una forma setencial de \(G\) y \(\beta\to\delta\in P\) entonces \(\alpha\delta\gamma\) es una forma setencial de \(G\).
\end{itemize}

\paragraph{Derivación directa en \(G\):} Si \(\alpha\beta\gamma\in(V_N\cup V_T)^*\) y \(\beta\to\delta\in P\) entonces \(\alpha\delta\gamma\) es una derivación directa de \(G\) de \(\alpha\beta\gamma\) y se denota como \(\alpha\beta\gamma\underset{G}{\implies}\alpha\delta\gamma\).
\begin{itemize}
  \item \(\overset{+}{\underset{G}{\implies}}\) es la clausura positiva.
  \item \(\overset{*}{\underset{G}{\implies}}\) es la clausura transitiva y reflexiva.
  \item \(\overset{k}{\underset{G}{\implies}}\) será la potencia \(k\)-ésima.
\end{itemize}

\paragraph{Lenguaje de una grámatica \(\mathcal{L}(G)\):} Es el conjunto de todas las cadenas de símbolos terminales que son formas setenciales de \(G\).

\[ \mathcal{L}(G) = \{ \alpha\in V_T^*:~S\overset{+}{\underset{G}{\implies}}\alpha\}\]

\subsubsection{Clasificación de grámaticas (Chomsky)}
\paragraph{Gramáticas regulares (tipo 3):} Son aquellas gramáticas que cumplen alguna de las siguientes condiciones:
\begin{itemize}
  \item Todas sus producciones son de la forma \(A\to aB\) ó \(A\to a\) ó \(A\to\lambda\) donde \(A,B\in V_N\) y \(a\in V_T\). En este caso se dice que es una gramática lineal a derecha.
  \item Todas sus producciones son de la forma \(A\to Ba\) ó \(A\to a\) ó \(A\to\lambda\) donde \(A,B\in V_N\) y \(a\in V_T\). En este caso se dice que es una gramática lineal a izquierda.
\end{itemize}

\paragraph{Gramáticas libres de contexto (tipo 2):} Son aquellas gramáticas en las que cada producción es de la forma \(A\to\alpha\) donde \(A\in V_N\) y \(\alpha\in(V_N\cup V_T)^*\).

De la definición anterior puede inferirse que toda grámatica regular es libre de contexto.

\paragraph{Gramáticas sensibles al contexto (tipo 1):} Son aquellas gramáticas en las que cada producción es de la forma \(\alpha\to\beta\) donde \(\alpha,\beta\in(V_N\cup V_T)^*\) y \(|\alpha|\leq |\beta|\).
Se puede inferir que toda gramática independiente del contexto que no posea regla borradoraas (es decir, que no posea producciones de la forma \(A\to\lambda\)) es sensible al contexto.

\paragraph{Gramáticas sin restricciones (tipo 0):} Son aquellas gramáticas que no poseen ninguna restricción sobre la forma de sus producciones. El conjunto de las grámaticas tipo 0 es el conjunto de todas las grámaticas y permite generar todos los lenguajes aceptados por una máquina de Turing.

\paragraph{Definición:} Un lenguaje generado por una grámatica tipo \(t\) es llamado \textbf{lenguaje tipo \(t\)}.
\newpage
\section{Grafos}
\subsection{Definiciones básicas}
Los grafos proporcionan una forma conveniente y flexible de representar problemas de la vida real que consideran una red como estructura subyacente. Esta red puede ser física (como instalaciones eléctticas) o abstractas (que modelan relaciones menos tangibles, como relaciones sociales y bases de datos).

Matemáticamente, un grafo \(G = (V,X)\) es un par de conjuntos, donde \(V\) es un conjunto de \textbf{puntos / nodos / vértices} y \(X\) es un subconjunot del conjunto de pares no ordenados de elementos distintos de \(V\). Los elementos de \(X\) se llamas \textbf{aristas, ejes o arcos}.

\begin{figure}[H]
	\begin{center}

		\begin{tikzpicture}
			\node[basicNode] (p) {\(p\)};
			\node[basicNode] (q) [right=of p] {\(q\)};
			\node[basicNode] (r) [below=of p] {\(r\)};
			\node[basicNode] (s) [below=of q] {\(s\)};

			\path
			(p) edge (q)
			(p) edge (r)
			(p) edge (s)
			(q) edge (s)
			(r) edge (s)
			;
		\end{tikzpicture}
	\end{center}
	\caption{\(G =([p,~q,~r,~s],~[(p,q),~(p,s),~(q,s),~(r,s)])\)
	}
\end{figure}

Dados \(v\) y \(w \in V\), si \(e=(v,w)\in X\) se dice que \(v\) y \(w\) son \textbf{adyacentes} y que \(e\) es \textbf{incidente} a \(v\) y a \(w\).

La definición de grafo no alcanza para modelar todas las situaciones posibles de una red. Por ejemplo, si se quisiese modelar la ruta de los aviones entre varias ciudades, deberiamos poder modelar varios vuelos entre dos ciudades:

\paragraph{Multigrafo:} Es un grafo en el que puede haber varias aristas entre el mismo par de nodos distintos.
\begin{figure}[H]
	\begin{center}
		\begin{tikzpicture}
			\node[basicNode] (p) {};
			\node[basicNode] (q) [right=of p] {};
			\path
			(p) edge[out=10, in=170] (q)
			(p) edge[out=-10, in=-170] (q)
			;
		\end{tikzpicture}
	\end{center}
	\caption{Multigrafo}
\end{figure}
\paragraph{Seudografo:} Es un grafo en el que puede haber varias aristas entre cada par de nodos y también puede haber aristas (\textit{loops}) que unan a un nodo con sí mismo.

\begin{figure}[H]
	\begin{center}
		\begin{tikzpicture}
			\tikzset{every loop/.style={}}
			\node[basicNode] (p) {};
			\path
			(p) edge[loop] (p);
		\end{tikzpicture}
	\end{center}
	\caption{Multigrafo}
\end{figure}

\paragraph{Notación:} \(n = |V|\) y \(m=|X|\)

\paragraph{Grado:} El grado \(d_G(v)\) de un nodo \(v\) es la cantidad de aristas incidentes a \(v\) en el grafo \(G\).

Notaremos \(\Delta(G)\) al máximo grado de los vértices de \(G\) y \(\delta(G)\) al mínimo.

\paragraph{Nota:} En un seudografo, un loop aporta 2 al grado del vértice.

\begin{theorem}
	La suma de los grados de los nodos de un grafo es igual a dos veces el número de aristas, es decir: \[\sum_{i=1}^{n}v_i = 2m\]
\end{theorem}

\begin{demo}
	Sea \(G = (V, X)\) un grafo de \(n\) nodos y \(m\) aristas. Haremos inducción en la cantidad de aristas:

	\paragraph{Caso base:} \(m = 1\).

	En este caso, el grafo \(G\) tiene sólo una arista que notamos \(e = (u,w)\). Entonces \(d(u) = d(w) = 1\) y \(d(v) = 0\) para todo \(v\in V\) tal que \(v\neq u,w\). Por lo tanto, \[\sum_{v\in V}d(v) = 2\] y \(2m = 2\), cumpliendose la propiedad.
	\paragraph{Paso inductivo:} Consideremos un grafo \(G=(V,X)\) con \(m\) aristas \((m > 1)\). Nuestra hipotesis inductiva es: \textit{En todo grafo \(G' = (V', X')\) con \(m'\) aristas \((m' < m)\), se cumple que } \[\sum_{v\in V'}{d_{G'}}(v) = 2m'\]

	Sea \(e = (u,w) \in X\), llamemos \(G'' = (V, X-\{e\})\) al grafo que resulta si le quitamos \(e\) a \(G\). Como la cantidad de aristas de \(G''\) es \(m-1\), \(G''\) cumple con la hipótesis inductiva. Entonces vale que
	\[\sum_{v\in V}{d_{G''}}(v) = 2(m-1)\]

	Como \(d_G(u) = d_G''(u) + 1\), \(d_G(w) = d_G''(w) + 1\) y \(d_G(x) = d_G''(x)~\forall~x\in V,~x\neq u,w\), obtenemos que: \[\sum_{v\in V}{d_{G}}(v) = \sum_{v\in V}{d_{G''}}(v) + 2 = 2(m-1) + 2 = 2m\]
\end{demo}

\begin{coro}
	La cantidad de vértices de grado impar de un grafo es par.
\end{coro}

\paragraph{Grafo completo:} Son grafos en los que todos sus vértices son adyacentes entre sí. Notaremos como \(K_n\) al grafo completo de \(n\) vértices.
\begin{center}
	\begin{tikzpicture}[graphStyle]
		\node[basicNode] (k11) at (0,0.75) {};
		\node[] (k1Label) [below=of k11, yshift=-0.23cm] {\(K_1\)};

		\node[basicNode] (k21) at (1,0.75) {};
		\node[basicNode] (k22) [right=of k21] {};
		\node[] (k2Label) [below=of k21, yshift=-0.23cm, xshift=0.6cm] {\(K_2\)};

		\node[basicNode] (k31) at (3,0.25) {};
		\node[basicNode] (k32) [right=of k31] {};
		\node[basicNode] (k33) [above=of k31, xshift=0.5cm] {};
		\node[] (k3Label) [below=of k31, yshift=0.27cm, xshift=0.6cm] {\(K_3\)};

		\node[basicNode] (k41) at (5,0.25) {};
		\node[basicNode] (k42) [right=of k41] {};
		\node[basicNode] (k43) [above=of k41] {};
		\node[basicNode] (k44) [above=of k42] {};
		\node[] (k4Label) [below=of k41, yshift=0.25cm, xshift=0.6cm] {\(K_4\)};

		\node[basicNode] (k51) at (7.5,0) {};
		\node[basicNode] (k52) [right=of k51] {};
		\node[basicNode] (k53) [above right=of k52, yshift=0.25cm, xshift=-0.25cm] {};
		\node[basicNode] (k54) [above left=of k53, xshift=-0.25cm] {};
		\node[basicNode] (k55) [below left=of k54, xshift=-0.25cm] {};
		\node[] (k5Label) [below=of k51, yshift=0.5cm, xshift=0.6cm] {\(K_5\)};

		\path
		(k21) edge (k22)

		(k31) edge (k32)
		(k31) edge (k33)
		(k32) edge (k33)

		(k41) edge (k42)
		(k41) edge (k43)
		(k41) edge (k44)
		(k42) edge (k43)
		(k42) edge (k44)
		(k43) edge (k44)

		(k51) edge (k52)
		(k51) edge (k53)
		(k51) edge (k54)
		(k51) edge (k55)
		(k52) edge (k53)
		(k52) edge (k54)
		(k52) edge (k55)
		(k53) edge (k54)
		(k53) edge (k55)
		(k54) edge (k55)
		;
	\end{tikzpicture}
\end{center}
\paragraph{Propiedad:} Un grafo completo de \(n\) nodos tiene $\frac{n(n-1)}{2}$ aristas

\paragraph{Grafo complemento:} Dado un grafo \(G=(V,X))\), su grafo complemento \(\bar{G} = (V, \bar{X})\) es el grafo con el mismo conjunto de vértices pero un par de vértices son adyacentes en $\bar{G}$ si y solo si no son adyacentes en $G$.
\begin{center}
	\begin{tikzpicture}[graphStyle]
		\node[basicNode] (v1) at (0,0) {};
		\node[basicNode] (v2) [above=of v1] {};
		\node[basicNode] (v3) [right=of v1] {};
		\node[basicNode] (v4) [right=of v2] {};
		\node[basicNode] (v5) [above right=of v3, yshift=-0.12cm] {};
		\node[] (glabel) [below right=of v1, xshift=-0.5cm] {Grafo \(G\)};

		\node[basicNode] (bv1) at (3,0) {};
		\node[basicNode] (bv2) [above=of bv1] {};
		\node[basicNode] (bv3) [right=of bv1] {};
		\node[basicNode] (bv4) [right=of bv2] {};
		\node[basicNode] (bv5) [above right=of bv3, yshift=-0.12cm] {};
		\node[] (bglabel) [below right=of bv1, xshift=-0.5cm] {Grafo \(\bar{G}\)};

		\path
		(v1) edge (v2)
		(v1) edge (v3)
		(v2) edge (v3)
		(v2) edge (v4)
		(v3) edge (v5)
		(v4) edge (v5)

		(bv1) edge (bv4)
		(bv1) edge[bend right=90] (bv5)
		(bv2) edge[bend left=90] (bv5)
		(bv3) edge (bv4)
		;;
	\end{tikzpicture}
\end{center}

\paragraph{Propiedad:} Si \(G\) tiene \(n\) vértices y \(m\) aristas, entonces: \[m_{\bar{G}} = \frac{n(n-1)}{2} - m\]

\subsubsection{Caminos y ciclos}
\paragraph{Camino:} Un camino en un grafo, es una secuencia alternada de vértices y aristas \[P = v_0e_1v_1e_2\dots v_{k-1}e_kv_k\] tal que un extremo de la arista \(e_i\) es \(v_{i-1}\) y el otro es \(v_i\) para \(i=1\dots k\).

\paragraph{Camino simple:} Es un camino que no pasa dos veces por el mismo vértice.

\paragraph{Sección:} La sección de un camino \(P = v_0e_1v_1e_2\dots v_{k-1}e_kv_k\) es una subsecuencia \[v_ie_{i+1}v_{i+1}e_{i+2}\dots v_{j-1}e_jv_j\] de términos consecutivos de \(P\), y lo notamos como \(P_{v_iv_j}\).

\paragraph{Circuito:} Es un camino que empieza y termina en el mismo vértice.

\paragraph{Circuito Simple:} Es un circuito de tres o más vértices que no pasa dos veces por el mismo vértices.

\begin{center}
	\begin{tikzpicture}[graphStyle]
		\node[basicNode] (v1) at (0,0) {\(v_1\)};
		\node[basicNode] (v2) [above=of v1] {\(v_2\)};
		\node[basicNode] (v3) [right=of v1] {\(v_3\)};
		\node[basicNode] (v4) [right=of v2] {\(v_4\)};
		\node[basicNode] (v5) [above right=of v3, yshift=-0.5cm] {\(v_5\)};
		\node[] (glabel) [below right=of v1, xshift=-1cm] {Camino no simple};
		\node[] (glabel1) [below=of glabel, yshift=0.75cm] {\red{\(P = v_2v_3v_1v_2v_4\)}};

		\node[basicNode] (bv1) at (5,0) {\(v_1\)};
		\node[basicNode] (bv2) [above=of bv1] {\(v_2\)};
		\node[basicNode] (bv3) [right=of bv1] {\(v_3\)};
		\node[basicNode] (bv4) [right=of bv2] {\(v_4\)};
		\node[basicNode] (bv5) [above right=of bv3, yshift=-0.5cm] {\(v_5\)};
		\node[] (bglabel) [below right=of bv1, xshift=-1cm] {Camino simple};
		\node[] (bglabel1) [below=of bglabel, yshift=0.75cm] {\red{\(P = v_1v_2v_5v_4\)}};

		\node[basicNode] (cv1) at (0,-5) {\(v_1\)};
		\node[basicNode] (cv2) [above=of cv1] {\(v_2\)};
		\node[basicNode] (cv3) [right=of cv1] {\(v_3\)};
		\node[basicNode] (cv4) [right=of cv2] {\(v_4\)};
		\node[basicNode] (cv5) [above right=of cv3, yshift=-0.5cm] {\(v_5\)};
		\node[] (cglabel) [below right=of cv1, xshift=-1.25cm] {Circuito no simple};
		\node[] (cglabel1) [below=of cglabel, yshift=0.75cm] {\red{\(P = v_1v_3v_2v_4v_5v_2v_1\)}};

		\node[basicNode] (dv1) at (5,-5) {\(v_1\)};
		\node[basicNode] (dv2) [above=of dv1] {\(v_2\)};
		\node[basicNode] (dv3) [right=of dv1] {\(v_3\)};
		\node[basicNode] (dv4) [right=of dv2] {\(v_4\)};
		\node[basicNode] (dv5) [above right=of dv3, yshift=-0.5cm] {\(v_5\)};
		\node[] (dglabel) [below right=of dv1, xshift=-1cm] {Circuito simple};
		\node[] (dglabel1) [below=of dglabel, yshift=0.75cm] {\red{\(P = v_2v_3v_5v_4v_2\)}};

		\path
		(v1) edge[redEdge]  (v2)
		(v1) edge[redEdge]  (v3)
		(v2) edge[redEdge] (v3)
		(v2) edge[redEdge]  (v4)
		(v2) edge (v5)
		(v3) edge (v5)
		(v4) edge (v5)

		(bv1) edge[red, line width=0.75mm]   (bv2)
		(bv1) edge (bv3)
		(bv2) edge (bv3)
		(bv2) edge (bv4)
		(bv2) edge[redEdge]   (bv5)
		(bv3) edge (bv5)
		(bv4) edge[redEdge]   (bv5)

		(cv1) edge[redEdge] (cv2)
		(cv1) edge[redEdge] (cv3)
		(cv2) edge[redEdge] (cv3)
		(cv2) edge[redEdge] (cv4)
		(cv2) edge[redEdge] (cv5)
		(cv3) edge (cv5)
		(cv4) edge[redEdge] (cv5)

		(dv1) edge (dv2)
		(dv1) edge (dv3)
		(dv2) edge[redEdge] (dv3)
		(dv2) edge[redEdge] (dv4)
		(dv2) edge (dv5)
		(dv3) edge[redEdge] (dv5)
		(dv4) edge[redEdge] (dv5)
		;
	\end{tikzpicture}
\end{center}

\paragraph{Longitud de un camino:} Dado un camino \(P\), su longitud \(l(P)\) es la cantidad de aristas que tiene.

\paragraph{Distancia:} La distancia \(d(v,w)\) entre dos vértices \(v\) y \(w\) se define como la longitud del camino más corto entre \(v\) y \(w\).
\begin{itemize}
	\item Si no existe camino entre \(v\) y \(w\) decimos que \(d(v,w) = \infty\).
	\item \(\forall~v\in V,~d(v,v) = 0\)
\end{itemize}

\begin{proposicion}
	Si un camino \(P\) entre \(v\) y \(w\) tiene longitud \(d(v,w)\) entonces \(P\) es un camino simple.
\end{proposicion}
\begin{demo}
	Demostración por el absurdo. Sea \(P = v\dots w\) un camino entre \(v\) y \(w\) con \(l(P) = d(v,w)\). Supongamos que \(P\) no es simple, es decir existe un vértice \(u\) que se repite en \(P\) (\(u\) podría llegar a ser \(v\) o \(w\)) entonces \(P = v\dots u \dots u \dots w\).

	Formemos ahora un camino \(P' = P_{vu}P_{uw}\), como \(P'\) no tiene todos los nodos que están en \(P_{uu}\) nos queda que \(l(P') < l(P) = d(v,w)\). Esto genera un absurdo porque por definición \(d(v,w)\) es la longitud del camino más corto entre \(v\) y \(w\).
\end{demo}

\begin{proposicion}
	La función de distancia cumple las siguientes propiedades para todo \(u,v,w\) pertenecientes a \(V\):
	\begin{enumerate}
		\item \(d(u,v)\geq 0\)
		\item \(d(u,v)=0 \iff u=v\)
		\item \(d(u,v) = d(v,u)\)
		\item \(d(u,w) \leq d(u,v) + d(v,w)\)
	\end{enumerate}
\end{proposicion}

\subsubsection{Subgrafos y Componentes Conexas}

\paragraph{Subgrafo:}
Dado un grafo \(G=(V_G, X_G)\), un subgrafo de \(G\) es un grafo \(H = (V_H, X_H)\) tal que \(V_H\subseteq V_G\) y \(X_H\subseteq X_G\cap (V_H\times V_H)\). Y notamos \(H\subseteq G\).

\begin{itemize}
	\item Si \(H \subseteq G\) y \(H\neq G\), entonces \(H\) es un \textbf{subgrafo propio} de \(G\) y notamos \(H \subset G\).
	\item \(H\) es un subgrafo generador de \(G\) si \(H \subseteq G\) y \(V_G = V_H\).
	\item Un subgrafo \(H=(V_H, X_H)\) de \(G=(V_G, X_G)\), es un \textbf{subgrafo inducido} si \(\forall~u,v\in V_H\) tal que \((u,v)\in X_G \Rightarrow (u,v)\in X_H\).
	\item Un subgrafo inducido de \(G=(V_G, X_G)\) por un conjunto de vértices \(V'\subseteq V_G\), se denota como \(G_{[V']}\).
\end{itemize}

\begin{center}
	\begin{tikzpicture}[graphStyle]
		\node[basicNode] (v1) at (0,0) {};
		\node[basicNode] (v2) [above left=of v1] {};
		\node[basicNode] (v3) [above right=of v1] {};
		\node[] (glabel) [below=of v1] {Grafo \(G\)};

		\node[basicNode] (bv1) at (3,0) {};
		\node[basicNode] (bv2) [above left=of bv1] {};
		\node[] (glabel) [below=of bv1, text width=2cm, text centered] {Sugrafo propio de \(G\)};

		\node[basicNode] (cv1) at (6,0) {};
		\node[basicNode] (cv2) [above left=of cv1] {};
		\node[basicNode] (cv3) [above right=of cv1] {};
		\node[] (glabel) [below=of cv1, text width=3cm, text centered] {Grafo que no es subgrafo de \(G\)};

		\node[basicNode] (dv1) at (9,0) {};
		\node[basicNode] (dv2) [above left=of dv1] {};
		\node[] (glabel) [below=of dv1, text width=2.5cm, text centered] {Subgrafo inducido de \(G\)};

		\path
		(v1) edge (v2)
		(v1) edge (v3)

		(cv2) edge (cv3)

		(dv1) edge (dv2)
		;
	\end{tikzpicture}
\end{center}
\paragraph{Grafo conexo:}
Un grafo se dice \textbf{conexo} si existe camino entre todo par de vértices.

\paragraph{Componente conexa:} Una componente conexa de un grafo \(G=(V_G, X_G)\) es un subgrafo conexo maximal de \(G\). Esto es un subgrafo \(H = (V_H, X_H)\) inducido de \(G\) tal que \(H\) es conexo y si tratamos de agregar cualquier vértice \(v\in V_G \backslash V_H\) entonces nos queda un grafo no conexo.


\begin{center}
	\begin{tikzpicture}[graphStyle]
		\node[basicNode] (v1) at (0,0) {};
		\node[basicNode] (v2) [above left=of v1] {};
		\node[basicNode] (v3) [above right=of v1] {};
		\node[basicNode] (v4) [right=of v3] {};
		\node[basicNode] (v5) [below=of v4] {};
		\node[] (glabel) [below=of v1, xshift=1cm] {Grafo conexo.};


		\node[basicNode] (bv1) at (5,0) {};
		\node[basicNode] (bv2) [above left=of bv1] {};
		\node[basicNode] (bv3) [above right=of bv1] {};
		\node[redNode] (bv4) [right=of bv3] {};
		\node[redNode] (bv5) [below=of bv4] {};
		\node[] (glabel) [below=of bv1, text width=4cm, xshift=1cm, text centered] {Grafo no conexo. Tiene dos componentes conexas (la \green{verde} y la \red{roja})};

		\path
		(v1) edge (v2)
		(v1) edge (v3)
		(v3) edge (v4)
		(v4) edge (v5)

		(bv1) edge (bv2)
		(bv1) edge (bv3)
		(bv4) edge (bv5)
		;
	\end{tikzpicture}
\end{center}

\subsection{Grafos bipartitos}
Un grafo \(G = (V,X)\) se dice \textbf{bipartito} si existe una partición \(V_1,V_2\) del conjunto de vértices \(V\) tal que:
\begin{itemize}
	\item \(V = V_1\cup V_2\)
	\item \(V_1\cap V_2 \neq\emptyset\)
	\item \(V_1\neq\emptyset\)
	\item \(V_2\neq\emptyset\)
	\item Todas las aristas de \(G\) tiene un extremo en \(V_1\) y otro en \(V_2\)
\end{itemize}

Un grafo bipartito con partición \(V_1,~V_2\) es \textbf{bipartito completo} si todo vértice en \(V_1\) es adyacente a todo vértice en \(V_2\).

\begin{center}
	\begin{tikzpicture}[graphStyle]
		\node[basicNode] (v1) at (0,0) {};
		\node[basicNode] (v2) [above left=of v1] {};
		\node[basicNode] (v3) [above right=of v1] {};
		\node[] (glabel) [below=of v1] {Grafo no bipartito};

		\node[redNode] (bv1) at (4,0) {};
		\node[basicNode] (bv2) [above=of bv1] {};
		\node[redNode] (bv3) [right=of bv2] {};
		\node[basicNode] (bv4) [below=of bv3] {};
		\node[basicNode] (bv5) [below left=of bv1] {};
		\node[redNode] (bv6) [above left=of bv2] {};
		\node[basicNode] (bv7) [above right=of bv3] {};
		\node[redNode] (bv8) [below right=of bv4] {};
		\node[] (bglabel) [below right=of bv5, xshift=-0.5cm] {Grafo bipartito};

		\node[basicNode] (cv1) at (8,-0.5) {};
		\node[basicNode] (cv2) [right=of cv1] {};
		\node[basicNode] (cv3) [right=of cv2] {};
		\node[basicNode] (cv4) [right=of cv3] {};
		\node[redNode] (cv5) [above=of cv1, yshift=1cm] {};
		\node[redNode] (cv6) [right=of cv5, xshift=.5cm] {};
		\node[redNode] (cv7) [above=of cv4, yshift=1cm] {};
		\node[] (cglabel) [below right=of cv1, xshift=-1cm] {Grafo bipartito completo};
		\path
		(v1) edge (v2)
		(v1) edge (v3)
		(v3) edge (v2)

		(bv1) edge (bv2)
		(bv1) edge (bv4)
		(bv1) edge (bv5)
		(bv2) edge (bv3)
		(bv2) edge (bv6)
		(bv3) edge (bv4)
		(bv3) edge (bv7)
		(bv4) edge (bv8)
		(bv5) edge (bv6)
		(bv5) edge (bv8)
		(bv6) edge (bv7)
		(bv7) edge (bv8)

		(cv1) edge (cv5)
		(cv1) edge (cv6)
		(cv1) edge (cv7)
		(cv2) edge (cv5)
		(cv2) edge (cv6)
		(cv2) edge (cv7)
		(cv3) edge (cv5)
		(cv3) edge (cv6)
		(cv3) edge (cv7)
		(cv4) edge (cv5)
		(cv4) edge (cv6)
		(cv4) edge (cv7)
		;
	\end{tikzpicture}
\end{center}

\begin{theorem}
	Un grafo \(G\) con dos o más vértices es bipartito si y solo si no tiene circuitos de longitud impar.
\end{theorem}
\begin{demo}
	Como un grafo es bipartito si y solo si cada una de sus componentes conexas es bipartita alcanza con demostrar el teorema para grafos conexos.
	\paragraph{\(\left.\Rightarrow\right) \)} Sea \(G\) un grafo conexo bipartito y \(V = (V_1,V_2)\) su bipartición.

	Si \(G\) no tiene circuitos entonces el teorema se cumple de manera trivial.

	Supongamos que \(G\) tiene circuitos y sea \(C = v_1v_2\dots v_kv_1\) un circuito de \(G\). Sin perdida de generalidad, supongamos que \(v_1\in V_1\). Como \((v_1,v_2) \in X\) entonces \(v_2\in V_2\)). En general, \(v_{2i + 1}\in V_1\) y \(v_{2i}\in V_2\). Como \(v_1\in V_1\) y \((v_k,v_1)\in X\), debe pasar \(v_k\in V_2\). Luego \(k = 2i\) para algún \(i\), lo que implica que \(l(C)\) es par.
\end{demo}
\begin{demoPart}

	\paragraph{\(\left.\Leftarrow\right)\)} Sea \(G\) un grafo conexo sin circuitos impares. Sea \(u\) cualquier vértice de \(V\). Definimos: \[V_1 = \{ v\in V~/~d(u,v) \text{ es par}\}\] \[V_2 = \{ v\in V~/~d(u,v) \text{ es impar}\}\]

	\(V_1\) y \(V_2\) definen una partición de \(V\) (ya que como \(G\) es conexo no hay vértices a distancia \(\infty\) de \(v\)). Tenemos que ver que definen una bipartición, es decir que no existe arista entre dos vértice de \(V_1\) y dos de \(V_2\). Hagamos esto por el absurdo:

	Supongamos que no es bipartición, es decir existen \(v,w\in V_1\) tales que \((v,w)\in X\). Si \(v = u\), entonces \(d(v,w) = 1\), pero esto no puede pasar por que \(d(u,w) \) es par. Lo mismo para \(w\), luego \(v\neq u,w\).

	Sea \(P\) un camino mínimo entre \(v\) y \(u\) y \(Q\) un camino mínimo entre \(v\) y \(w\). Como \(u,w\in V_1\), \(P\) y \(Q\) tienen longitud par.

	Sea \(z\in V\) el último nodo en el que \(P\) y \(Q\) se cruzan (podría pasar que \(z = u\)). Como \(P\) y \(Q\) definen las distancias a \(v\) y \(w\) respectivamente desde \(u\), entonces \(P_{uz}\) y \(Q_{uz}\) tienen que ser caminos mínimos. Osea que \[l(P_{uz}) = l(Q_{uz}) = d(u,z)\].
	Entonces \(l(P_{zv})\) y \(l(Q_{zw})\) tienen igual paridad. Definamos \[C = P_{zv}(v,w)Q_{wz}\]
	Entonces \(l(C) = l(P_{zv}) + l(Q_{wz}) + 1\) que es una longitud impar. Absurdo, partiamos de la supocisión de que \(G\) no tenia circuitos de longitud impar.
\end{demoPart}

\subsection{Representación de grafos}
\subsubsection{Matriz de adyacencia de un grafo}
Dado un grafo \(G\), se define su \textbf{matriz de adyacencia} \(A\in\reales^{n\times n}\), \(A = [a_{ij}]\) como:

\[a_{ij} = \begin{cases}
		1 & \text{ si } G \text{ tiene una arista entre } v_i \text{ y } v_j \\
		0 & \text{ si no }
	\end{cases}
\]

\begin{proposicion}
	Si \(A\) es la matriz de adyacencia del grafo \(G\), entonces:
	\begin{itemize}
		\item La suma de los elementos de la columna (o fila) \(i\) de \(A\) es igual a \(d(v_i)\).
		\item Los elementos de la diagonal de \(A^2\) indican los grados de los vértices: \(a_{ii}^2 = d(v_i)\).
	\end{itemize}
\end{proposicion}

Para los seudografos, se generaliza la definición dada de la seguiente manera:

\[a_{ij} = \begin{cases}
		\text{ cantidad de aristas }(v_i,v_j) & \text{ si } i \neq j \\
		\text{ cantidad de loops sobre } v_i  & \text{ si } i = j
	\end{cases}
\]

\subsubsection{Matriz de incidencia de un grafo}
Dado un grafo \(G\), se define su \textbf{matriz de incidencia} \(B\in\reales^{m\times n}\) con \(B = [b_{ij}]\) como:

\[b_{ij} = \begin{cases}
		1 & \text{ si la arista } i \text{ es incidente al vértice } v_j \\
		0 & \text{ sino }
	\end{cases}\]

\begin{proposicion}
	Si \(B\) es la matriz de incidencia del grafo \(G\), entonces:
	\begin{itemize}
		\item La suma de los elementos de cada fila es igual a 2.
		\item La suma de los elementos de la \(j\)-ésima columna es igual a \(d(v_j)\).
	\end{itemize}
\end{proposicion}

Para los pseudografos, se generaliza la definición de la siguiente forma:

\[b_{ij} = \begin{cases}
		2 & \text{ si la arista } i \text{ es loop sobre el vertice } v_j             \\
		1 & \text{ si la arista } i \text{ no es loop e incide sobre el vértice } v_j \\
		0 & \text{ sino }
	\end{cases}\]

\subsection{Digrafos}
\paragraph{Grafo dirigido o digrafo:} Es un par de conjuntos \(G = (V,X)\) donde \(V\) es el conjunto de nodos y \(X\) es un subconjunto del conjunto de pares \textbf{ordenados} de elementos distintos de \(V\). A los elementos de \(X\) los llamaremos \textbf{arcos}.

Dado un arco \(e=(u,w)\) llamaremos al primer elemento (\(u\)) \textbf{cola} de \(e\) y al segundo (\(w\)), \textbf{cabeza} de \(e\).

\begin{center}
	\begin{tikzpicture}[graphStyle]
		\node[basicNode] (v1) at (0,0) {};
		\node[basicNode] (v2) [above=of v1] {};
		\node[basicNode] (v3) [right=of v1] {};
		\node[basicNode] (v4) [right=of v2] {};
		\node[basicNode] (v5) [above right=of v3, yshift=-0.12cm] {};
		\node[] (glabel) [below right=of v1, xshift=-0.75cm] {Digrafo \(G\)};

		\path
		(v1) edge[-stealth] (v2)
		(v1) edge[-stealth] (v3)
		(v2) edge[-stealth, bend right=15] (v3)
		(v2) edge[-stealth] (v4)
		(v3) edge[-stealth, bend right=15] (v2)
		(v3) edge[-stealth] (v5)
		(v4) edge[-stealth] (v5)
		;
	\end{tikzpicture}
\end{center}

\paragraph{Grado de entrada:} \(d_{in}(v)\) de un vértice \(v\) de un digrafo es la cantidad de arcos que llegan a \(v\). Es decir, la cantidad de arcos que tienen como cabeza a \(v\).

\paragraph{Grado de entrada:} \(d_{out}(v)\) de un nodo \(v\) de un digrafo es la cantidad de arcos que salen de \(v\). Es decir, la cantidad de arcos que tiene a \(v\) como cola.

\paragraph{Grafo subyacente:} El grafo subyacente de un digrafo \(G\) es el grafo \(G^S\) que resulta de remover las direcciones de sus arcos (si para un par de vértices hay arcos en ambas direcciones, sólo se coloca una arista entre ellos).

\begin{center}
	\begin{tikzpicture}[graphStyle]
		\node[basicNode] (v1) at (0,0) {};
		\node[basicNode] (v2) [above=of v1] {};
		\node[basicNode] (v3) [right=of v1] {};
		\node[basicNode] (v4) [right=of v2] {};
		\node[basicNode] (v5) [above right=of v3, yshift=-0.12cm] {};
		\node[] (glabel) [below right=of v1, xshift=-0.75cm] {Digrafo \(G\)};

		\node[basicNode] (sv1) at (3,0) {};
		\node[basicNode] (sv2) [above=of sv1] {};
		\node[basicNode] (sv3) [right=of sv1] {};
		\node[basicNode] (sv4) [right=of sv2] {};
		\node[basicNode] (sv5) [above right=of sv3, yshift=-0.12cm] {};
		\node[] (sglabel) [below right=of sv1, xshift=-0.75cm] {Grafo \(G^S\)};

		\path
		(v1) edge[-stealth] (v2)
		(v1) edge[-stealth] (v3)
		(v2) edge[-stealth, bend right=15] (v3)
		(v2) edge[-stealth] (v4)
		(v3) edge[-stealth, bend right=15] (v2)
		(v3) edge[-stealth] (v5)
		(v4) edge[-stealth] (v5)

		(sv1) edge (sv2)
		(sv1) edge (sv3)
		(sv2) edge (sv3)
		(sv2) edge (sv4)
		(sv3) edge (sv5)
		(sv4) edge (sv5)
		;
	\end{tikzpicture}
\end{center}

\paragraph{Matriz de adyacencia:} de un digrafo \(G\), \(A\in\reales^{n\times n}\), \(A = [a_{ij}]\) se define como:

\[a_ij = \begin{cases}
		1 & \text{ si } G \text{ tiene un arco } v_i a v_j \\
		0 & \text{ si no }                                 \\
	\end{cases}\]

\begin{proposicion}
	Si \(A\) es la matriz de adyacencia del digrafo \(G\), entonces:
	\begin{itemize}
		\item La suma de los elementos de la fila \(i\) de \(A\) es igual a \(d_{out}(v_i)\).
		\item La suma de los elementos de la columna \(i\) de \(A\) es igual a \(d_{in}(v_i)\).
	\end{itemize}
\end{proposicion}

\paragraph{Matriz de incidencia:} de un digrafo \(G\), \(B\in\reales^{m\times n}\), \(B = [b_{ij}]\) se define como:

\[b_ij = \begin{cases}
		1  & \text{ si } v_j \text{ es cabeza del arco } i \\
		-1 & \text{ si } v_j \text{ es cola del arco } i   \\
		0  & \text{ si no }                                \\
	\end{cases}\]

\begin{proposicion}
	Si \(B\) es la matriz de incidencia del digrafo \(G\), entonces la suma de los elementos de cada fila es igual a cero.
\end{proposicion}

\paragraph{Camino orientado:} Es una sucesión de arcos \(e_1e_2\dots e_k\) tal que el primer elemento del arco \(e_i\) coincide con el segundo de \(e_{i-1}\) y el segundo elemento de \(e_i\) con el primero de \(e_{i+1}\) con \(i = 2,\dots,k-1\)

\paragraph{Grafo fuertemente conexo:} Es un digrafo \(G\) tal que para todo par de vértices \(u,v\in 	V_G\) existe un camino orientado de \(u\) a \(v\).


\newpage
\section{{Árboles}}
\paragraph{Árbol:} Grafo conexo sin circuitos simples.

\begin{center}
	\begin{tikzpicture}[graphStyle]
	\node[basicNode] (v1) at (0,0) {};
	\node[] (glabel) [below=of v1, xshift=0.5] {Árbol trivial};
	
	\node[basicNode] (sv1) at (3,-0.25) {};
	\node[basicNode] (sv2) [above=of sv1] {};
	\node[basicNode] (sv3) [right=of sv1] {};
	\node[basicNode] (sv4) [right=of sv2] {};
	\node[basicNode] (sv5) [above right=of sv3, yshift=-0.12cm] {};
	\node[] (sglabel) [below right=of sv1, xshift=-0.75cm] {Árbol \(T\)};
	
	\path
		(sv1) edge (sv2)
		(sv2) edge (sv3)
		(sv2) edge (sv4)
		(sv3) edge (sv2)
		(sv3) edge (sv5)
	;
	\end{tikzpicture}
\end{center}

\paragraph{Puente:} Una arista \(e\) de un grafo \(G\) tal que \(G - e\) tiene más componentes conexas que \(G\).

\begin{center}
	\begin{tikzpicture}[graphStyle]
	\node[basicNode] (v1) at (0,0) {\(v_1\)};
	\node[basicNode] (v2) [below=of v1, xshift=-0.75cm] {\(v_2\)};
	\node[basicNode] (v3) [below=of v2] {\(v_3\)};	
	\node[basicNode] (v4) [right=of v2] {\(v_4\)};
	\node[basicNode] (v5) [below=of v4] {\(v_5\)};
	\node[basicNode] (v6) [left=of v2, yshift=-0.75cm] {\(v_6\)};	
	\node[basicNode] (v7) [right=of v4, yshift=-0.75cm] {\(v_7\)};
	\node[basicNode] (v8) [above right=of v4]  {\(v_8\)};
	\node[] (glabel) [below=of v3, xshift=0.95cm] {\red{Puentes: \((v_1,v_8), (v_2, v_4), (v_3, v_5)\)}};
	
	\path
	(v1) edge[redEdge] (v8)
	(v2) edge (v3)
	(v2) edge[redEdge]  (v4)
	(v2) edge (v6)
	(v3) edge[redEdge]  (v5)
	(v3) edge (v6)
	(v4) edge (v7)
	(v4) edge (v8)
	(v7) edge (v8)
	;
	\end{tikzpicture}
\end{center}


\begin{lema}\label{unionCaminosSimplesEsCircuito}
	La unión de dos caminos simples distintos entre dos vértices contiene un circuito simple.
\end{lema}

\begin{lema}\label{conexomenosesiiecircuitosimple}
	Sea \(G =(V, X)\) un grafo conexo y \(e\in X\). \(G-e = (V, X\backslash\{e\})\) es conexo si y solo si \(e\) pertenece a un circuito simple de \(G\).
	
	En otras palabras: Una arista \(e\in X\) es puente si y solo si \(e\) no pertence a ningún circuito simple de \(G\)
\end{lema}

\begin{demo}[]
	\paragraph{\(\Rightarrow\))} Sea \(e =(u,v)\in X\). Por hipotesis \(G-e\) es conexo. Entonces existe un camino simple \(P_{uw}\) entre \(u\) y \(w\) en \(G - e\) (que no usa a \(e\)). Luego podemos definir a \(C = P_{uw} + e\) como un circuito simple de \(G\) que contiene a \(e\).
	
	\paragraph{\(\Leftarrow\) )} Sea \(C\) un circuito simple de \(G\) que contiene a \(e = (u,w)\). Entonces podemos partir a \(C\) en la arista \(e\) y un camino simple \(P_{uw})\) entre \(u\) y \(w\) (que no contiene a \(e\)).

	Como \(G\) es conexo, hay camino entre todo par de vértices. Si esos caminos no una a \(e\), entonces siguen estando en \(G-e\).
\end{demo}
\begin{demoPart} 
	Si un camino de \(G\) usa \(e\), en \(G-e\) hay un camino alternativo que es cambiando a \(e\) por \(P_{uw}\) en \(Q\). Entonces sigue habiendo camino entre todo par de vértices en \(G-e\). Osea \(G-e\) es conexo.
\end{demoPart}

\begin{theorem}\label{equivalenciasArbol}
	Dado un grafo \(G=(V,X)\), son equivalentes:
	\begin{enumerate}
		\item \(G\) es un árbol.
		\item \(G\) es un grafo sin circuitos simples y para toda arista \(e\) tal que \(e\notin X\), \(G+e = (V, X\cup\{e\})\) tiene exactamente un circuito simple. Además, ese circuito contiene a \(e\).
		\item Existe exactamente un camino simple entre todo par de vértices.
		\item \(G\) es conexo pero si se quita cualquier arista a \(G\) queda un grafo no conexo (toda arista es puente).
	\end{enumerate}
\end{theorem}

\begin{demo}
	Vamos a demostrar \(1\Rightarrow 2\), \(2 \Rightarrow 3\), \(3 \Rightarrow 4\) y \(4 \Rightarrow 1\).

\paragraph{1 \(\Rightarrow\) 2)} \(G\) es árbol, es decir, conexo y sin circuitos. Sea \(e=(u,w)\notin X\). Como \(G\) es conexo existe un camino simple \(P_{uw}\) en \(G\) entre \(u\) y \(w\). Entonces \(P_{uw} + e\) es un circuito simple de \(G + e\).
	
	Ahora nos falta ver que no se puede haber generado más de un circuito simple. Vamos a hacerlo por el absurdo. Como sabemos que antes de agregar \(e\), \(G\) no tenía circuitos, todo ciruito simple de \(G + e\) tiene que contener a \(e\), ya que de lo contrario ese circuito estaría en \(G\).
	
	Entonces supongamos que existen dos circuitos simples \(C\) y \(C'\) en \(G + e\) que contienen \(e\). Podemos partir estos circuitos en \[ C = P_{uw} + e\] \[C'	= P'_{uw} + e\]
	
	\(P_{uw}\) y \(P'_{uw}\) son dos caminos simples distintos entre \(u\) y \(v\) en \(G\). Por el Lema \ref{unionCaminosSimplesEsCircuito}, la unión de estos dos cáminos en \(G\) es un circuito simple. Esto es absurdo ya que \(G\) es un árbol.

	\paragraph{2 \(\Rightarrow\) 3)} Primero veamos que \(G)\) es conexo, es decir que hay camino entre todo par de vértices \(u\) y \(v\). Si \((u,w)\in X_G\), el camino es la arista \((u,w)\). Si \(u\) y \(v\) no son adyacentes en \(G\), por hipotesis, \(G+(u,v)\) contiene exactamente un circuito simple \(C\) que contiene a \(u,w\). Podemos partir ese circuito en \(C = P_{uw} + (u,w)\) y \(P_{uw}\) es un camino simple entre \(u,v\).
\end{demo}
\begin{demoPart}	
	\paragraph{3 \(\Rightarrow\) 4)} Por hipotesis, \(G\) es conexo. Supongamos que existe \(e=(u,v)\in X\) tal que \(G - e\) es conexo. Por Lema \ref{conexomenosesiiecircuitosimple}, \(e\) pertenece a un circuito simple \(C\) de \(G\). Partir a \(C = P_{u,v} + (u,v)\), entonces \(P_{uv}]\) y \((u,v)\) son dos caminos simples distintos entre \(u\) y \(v\), contradiciento \(3\).
	
	\paragraph{4 \(\Rightarrow\) 1)} Por hipotesis, \(G\) es conexo. Por contradicción, supongamos que no es un árbol, osea que tiene un circuito \(C\). Sea \(e=(u,w)\)) una arista de ese circuito. Por Lema \ref{conexomenosesiiecircuitosimple}, \(G - e\) es conexo, pero esto contradice \(4\). Luego \(G\) es conexo y sin circuitos, osea es un árbol.
\end{demoPart}

\paragraph{Hoja:} Una hoja es un nodo de grado 1.

\begin{lema}\label{arboldoshojas}
	Todo árbol no trivial \(T\) (de al menos dos vértices) tiene al menos dos hojas.
\end{lema}
\begin{demo}
	Sea \(P: v_1\dots v_k\) un camino simple máximal (no extendible por sus extremos) en el árbol \(T\). Veamos que \(v_1\)) y \(v_k\) son hojas. 	
	
	Supongamos que \(d(v_1) > 1\). Entonces \(v_1\) tiene otro vértice adyacente
	 \(w\) además de \(v_2\) y \(w \neq v_2\). Como \(P\) es máximal, \(w\)
	  necesariamente ya tiene que estar en \(P\) (sino podriamos agregarlo y seguir extiendiendo el
	 camino).
	 Luego, \(P_{v_1,w} + (v_1, w)\) es un circuito, contradiciento que \(T\) es un árbol.
	Se puede utilizar el mismo razonamiento para deducir que \(d(v_k) = 1\).
\end{demo}

\begin{lema}\label{arbolcantEjes}
	Sea \(G = (V, X)\) un árbol. Entonces \(m = n - 1\).
\end{lema}
\begin{demo}
	Vamos a hacer inducción en la cantidad de vértices.
	\paragraph{Caso base (\(n = 1\)):} Se cumple por que \(n = 1\) y \(m = 0\)
	\paragraph{Caso inductivo:} Sea \(T = (V, X)\) un árbol con \(k\) vértices (\(k > 1\)). Nuestra hipotesis inductiva es: \[\forall~T' ~/~ n_{T'} \leq k \implies m_{T'} = n_{T'} - 1\]
	
	
	Por lema \ref{arboldoshojas}, \(T\) tiene al menos una hoja \(u\). Definamos \(T - u = (V \backslash \{u\},~X\backslash \{(u,v)\). \(T - u\) es conexo y no tiene circuitos, osea que es un árbol de \(k-1\). Por hipotesis inductiva, \(T - u\) tiene \(k-2\) aristas. Como \(d(u) = 1\), \(T\) tiene una arista más. Luego \(T\) tiene \(k-1\) aristas.
\end{demo}

\paragraph{Bosque:} Es un grafo sin circuitos simples.
\begin{coro}\label{bosqueCantEjes}
	Sea \(G = (V,X)\) sin circuitos simples y \(c\) componentes conexas. Entonces \(m = n - c\).
\end{coro}
\begin{demo}
Para \(i=1,\dots,c\), sea \(n_i\) la cantidad de vértices de la componente \(i\) y \(m_i\) la cantidad de aristas.

Como cada componente conexa de \(G\) es un árbol, podemos aplicar el lema \ref{arbolcantEjes}. Entonces \(m_i = n_i - 1~\forall~i=1\dots c\). Sumando, obtenemos que \[ m = \sum_{i=1}^{c}m_i = sum_{i=1}^{c}n_i - 1 = n - c\]
\end{demo}

\begin{coro}\label{bosqueNC}
	Sean \(G = (V, X)\) un bosque con \(c\) componentes conexas. Entonces \(m \geq n -c\)
\end{coro}
\begin{demo}
	Si \(G\) tiene circuitos, removerlos sacando una arista por vez hasta que el grafo resultante \(\hat{G}\) sea sin circuitos. Entonces, por Lema \ref{bosqueCantEjes}, \(\hat{m} = n - \hat{c}\)
\end{demo}

\begin{theorem}
	Dado un grafo \(G\) son equivalentes:
	\begin{enumerate}
		\item \(G\) es un árbol
		\item \(G\) es un grafo sin circuitos simples y \(m = n-1\)
		\item \(G\) es conexo y \(m = n - 1\)
	\end{enumerate}
\end{theorem}
\begin{demo}
	\paragraph{1 \(\Rightarrow\) 2)} Como \(G\) es árbol, \(G\) no tiene circuitos y \(m = n - 1\) por Lema \ref{arbolcantEjes}
	\paragraph{2 \(\Rightarrow\) 3)} Sea \(c\) lacantidad de componentes conexas de \(G\). Por corolario \ref{bosqueNC} , \(n - 1 = m = n - c\).
	\paragraph{3 \(\Rightarrow\)  1)} Por contradicción, supongamos que \(G\) tiene un circuito simple. Sea \(e = (v,w)\) una arista del circuito. Entonces, por Lema \ref{unionCaminosSimplesEsCircuito}, \(G' = G - e\) es conexo y \(m' = n - 2\), contradiciendo el corolario \ref{bosqueCantEjes}. Luego, \(G\) no puede tener circuitos simples. \(G\) es árbol.
\end{demo}

\subsection{Árboles enraizados}
Un \textbf{árbol enraizado} es un árbol qeue tiene un vértice distinguido que llamamos \textbf{raíz}. Explicitamente queda definido un árbol dirigido, considerando caminos orientados desde la raíz al resto de los vértices. 

\begin{itemize}
	\item Los vértices \textbf{internos} de un árbol son aquellos que no son ni hojas ni la raíz.
	\item El \textbf{nivel} de un vértice de un árbol con raíz en la distancia de la raíz a ese vértice.
	\item Decimos que dos vértices adyacentes tienen \textbf{relación parde-hijo}, siendo el padre el vértice de menor nivel.
	\item La \textbf{altura} \(h\) de un árbol con raíz es la distancia desde la raíz al vértice más lejano.
	\item Un árbol se dice (exactamente) \(m-ario\) si todos sus vértices,salvo las hojas y las raíz tienen grado (exactamente) a lo sumo \(m + 1\) y la raíz (exactamanete) a lo sumo \(m\).
	\item Un árbol se dice \textbf{balanceado} si todas sus hojas están a nivel \(h\) ó \(h - 1\).
	\item Un árbol se dice \textbf{balanceado completo} si todas sus hojas están a nivel \(h\).
\end{itemize}

\begin{theorem}
	Sea \(T\) un árbol \(m\)-ario de altura \(h\) con \(l\) hojas entonces:
	\begin{enumerate}
		\item \(T\) tiene a lo sumo \(m^h\) hojas.
		\item \(h \geq \lceil \log_m{l}\rceil\)
		\item Si \(T\) es un árbol exáctamente m-ario balancaeando completo entonces \(h = \lceil \log_m{l}\rceil\)
	\end{enumerate}
\end{theorem}
\begin{demo}[DE 1]
	Lo hacemos por inducción en la altura del árbol:
	\paragraph{Caso base (\(h = 1\)):} Por definición, en un árbol $m$, cada nodo tiene a lo sumo \(m\) hijos. Si tomamos la raíz, entonces \(d(r) \leq m\). Y como el árbol es de altura \(1\), todos sus hijos son hojas entonces: \(\# hojas = d(r) \leq m = m^1\)
	\vspace*{1mm}
		\begin{center}
		\begin{tikzpicture}[graphStyle]
			\node[basicNode, scale=0.37] (v1) at (0,0) {};
			\node[basicNode, scale=0.37] (v2) [below left=of v1] {};
			\node[basicNode, scale=0.37] (v3) [below right=of v1] {};
			\node[basicNode, scale=0.37] (v4) [below=of v1] {};
			\node[emptyDemoNode] (r) [above=of v1, yshift=-0.75cm] {r};
			\node[emptyDemoNode] (l1) [left=of v1]{};
			\node[emptyDemoNode] (r1) [right=of v1]{};
			\node[emptyDemoNode] (lr1) [below right=of r1, yshift=0.75cm, xshift=-0.65cm]{\(\leq m\)};

			\path
			(v1) edge (v2)
			(v1) edge (v3)
			(v1) edge (v4)
			(l1) edge[dashed,bend right=45] (r1)
			;
		\end{tikzpicture}
	\end{center}
	\paragraph{Paso inductivo:} Sea \(T = (V,X)\) un árbol \(m\)-ario de altura \(h > 1\). Nuestra hipotesis inductiva es:
	\begin{center}
		Todo árbol \(m\)-ario de altura \(h' < h\) tiene a lo sumo \(m^{h'}\) hojas.
	\end{center}
	Sea \(r\) la raíz de \(T\), como \(T\) es un árbol \(m\)-ario, \(d(r) = k \leq m\). Sean \(T_1,\dots,T_k\) las componentes conexas de \(T-r\):
			\begin{center}
		\begin{tikzpicture}[graphStyle]
		\node[basicNode, scale=0.37] (v1) at (0,0) {};
		\node[basicNode, scale=0.37] (v2) [below left=of v1] {};
		\node[basicNode, scale=0.37] (v3) [below right=of v1] {};
		\node[basicNode, scale=0.37] (v4) [below=of v1] {};
		\node[emptyDemoNode] (r) [above=of v1, yshift=-0.75cm] {r};
		\node[emptyDemoNode] (l1) [left=of v1]{};
		\node[emptyDemoNode] (r1) [right=of v1]{};
		\node[emptyDemoNode] (lr1) [below right=of r1, yshift=0.75cm, xshift=-0.65cm]{\(\leq m\)};
		
		\node[emptyDemoNode] (t1r) [below right=of v2, xshift=-0.5cm]{};
		\node[emptyDemoNode] (t1l) [below left=of v2, xshift=0.5cm]{};
		\node[emptyDemoNode, inner ysep=0.25cm] (t1b) [below=of v2]{\(T_1\)};
		
		\node[emptyDemoNode] (t2r) [below right=of v4, xshift=-0.5cm]{};
		\node[emptyDemoNode] (t2l) [below left=of v4, xshift=0.5cm]{};
		\node[emptyDemoNode, inner ysep=0.25cm] (t2b) [below=of v4, yshift=0.25cm]{\(\dots\)};
		
		\node[emptyDemoNode] (t3r) [below right=of v3, xshift=-0.5cm]{};
		\node[emptyDemoNode] (t3l) [below left=of v3, xshift=0.5cm]{};
		\node[emptyDemoNode, inner ysep=0.25cm] (t3b) [below=of v3]{\(T_k\)};
		
		\path
		(v1) edge (v2)
		(v1) edge (v3)
		(v1) edge (v4)
		(l1) edge[dashed,bend right=45] (r1)
		(v2) edge[dashed] (t1r) 
		(v2) edge[dashed] (t1l) 
		(v2) edge[dashed] (t1b) 
		
		(v4) edge[dashed] (t2r) 
		(v4) edge[dashed] (t2l) 
		(v4) edge[dashed] (t2b) 
		
		(v3) edge[dashed] (t3r) 
		(v3) edge[dashed] (t3l) 
		(v3) edge[dashed] (t3b) 
		;
		\end{tikzpicture}
	\end{center}
\end{demo}
\begin{demoPart}
	Cada \(T_i\) tiene altura \(h_i \leq h - 1\), osea que por la HI vale que \(\#hojas(T_i) \leq m^{h_i} \leq m^{h-1}\). Además, de que todas las hojas de cada \(T_i\) son hojas \(T\). 
	
	Si hay algun \(T_i\) cuya altura sea \(0\), entonces no tiene hojas pero \(T_i\) era un hoja de \(T\), sin embargo sigue valiendo que \(0 \leq m^0 \leq m^{h-1}\).

	Entonces:
	\begin{align*}
	\#hojas(T) &= \sum_{\substack{i=0 \\ h_i > 0}}^{k}\underbrace{\#hojas(T_i)}_{\leq m^{h_i}} + \sum_{\substack{i=0 \\ h_i = 0}}^{k}\hspace*{5mm}\underbrace{1}_{\leq m^{h_1}} \\ &\leq \sum_{i=0}^{k} m^{h_i} \leq \sum_{i=0}^{k} m^{h - 1} \leq \sum_{i=0}^{m} m^{h - 1} \\ &\leq m\times m^{h-1} \leq m^h
	\end{align*}
\end{demoPart}

\begin{demo}[DE 2]
	Por 1, sabemos que \(l \leq m^h \implies \log_m l \leq h\) y como \(h\) es entero \(\lceil\log_m l\rceil \leq h\)
\end{demo}
\begin{demo}[DE 3]	
	Si \(T\) es un árbol exactamente \(m\)-ario balanceado completo, entonces los \(T_1, \dots, T_k\) de la demostración del punto 1 son también árboles \(m\)-arios balanceados completos (por lo que cumplen la hipotesis inductiva) y los pasos son todas igualdades.
\end{demo}

\subsection{Árboles generadores}
Un \textbf{árbol generador} (\(AG\) de un grafo \(G\) es un subgrafo generador (que tiene el mismo conjunto de vértices ) de \(G\) que es árbol.
\begin{center}
	\begin{tikzpicture}[graphStyle]
	\node[basicNode] (v1) at (0,0) {};
	\node[basicNode] (v2) [above=of v1] {};
	\node[basicNode] (v3) [right=of v1] {};
	\node[basicNode] (v4) [right=of v2] {};
	\node[basicNode] (v5) [above right=of v3, yshift=-0.12cm] {};
	\node[] (glabel) [below right=of v1, xshift=-2cm] {Grafo \(G\) y un AG(\(G\)) \red{\(T\)}};
		
	\path
	(v1) edge[redEdge] (v2)
	(v1) edge[redEdge] (v3)
	(v2) edge (v3)
	(v2) edge[redEdge] (v4)
	(v3) edge[redEdge] (v5)
	(v4) edge (v5)
	;
	\end{tikzpicture}
\end{center}

\begin{theorem}\label{arbolesGeneradores}
	Sea \(G=(V,X)\) un grafo conexo:
	\begin{enumerate}
		\item \(G\) tiene (al menos) un árbol generador.
		\item \(G\) tiene un único árbol generador si y solo si \(G\) es un árbol.
		\item Sea \(T = (V, X_T)\) y \(e\in X \backslash X_T \). Sea \(f\neq e\) una arista del circuito que se genera cuando se agrega \(e\) a \(T\). Entonces \(T + e - f\) es un árbol generador de \(G\).	
	\end{enumerate}
\end{theorem}
\begin{demo}[DE 1]
	Sea \(G=(V,X)\) un grafo conexo, podemos construir un árbol generador \(T\) de la siguiente manera:
\begin{algorithmic}
	\State Definimos \(T = (V, X_T)\) con \(X_T = X\). 
	\While{\(T\) tenga algún circuito}
		\State Seleccionamos \(e\in X_T\) que pertenezca a un circuito de \(T\).
		\State \(T = T - e\)
	\EndWhile
\end{algorithmic}
Como todas las aristas removidas pertenecen a un circuito (no son puentes), cuando termina el procedimiento \(T\) es un subgrafo conexo generador de \(G\) (por lema \ref{conexomenosesiiecircuitosimple})  y todas sus aristas son puente. Entonces, por el teorema \ref{equivalenciasArbol}, \(T\) es un árbol.
\end{demo}

\begin{demo}[DE 2]
	\paragraph{\(\Rightarrow)\)} \(G\) tiene un único árbol generador. Supongamos que no es un árbol, entonces tiene por lo menos un circuito simple \(C\) (supongamos que tiene exactamente uno, sin perdida de generalidad). 
	
	Sean \(e\) y \(f\) dos aristas de \(C\), entonces siguiendo el procedimiento de la demostración anterior podemos formar dos árboles generadores distintos de \(G\): \(T_1 = G - e\) y \(T_2 = G -f\), si elegimos sacar \(e\) o \(f\), respectivamente.
	
	En ambos casos, sabemos que son árboles porque \(C\) era el único ciclo simple de \(G\) y quitarle una arista lo rompe. Luego, llegamos a un absurdo que proviene de suponer que \(G\) no era un árbol.
	
	\paragraph{\(\Leftarrow\)} \(G\) es un árbol.Por teorema \ref{equivalenciasArbol}, todas sus aristas son puente, es decir que cualquier arista que \(e\) saquemos hace que \(G - e\) deje de ser conexo. Luego, \(G\) es un árbol generador de si mismo.
\end{demo}

\begin{demo}[DE 3]
	Como \(T\) es un árbol generador de \(G\), entonces por el teorema \ref{equivalenciasArbol}, sabemos que \(T + e\) tiene exactamente un circuito \(C\). Sea \(e\neq f\) una arista de \(C\), por lema \ref{conexomenosesiiecircuitosimple}, si se quita una arista de un circuito el grafo sigue siendo conexo, es decir \(T + e - f\) es conexo.
	
	Entonces \(T + e - f\) es un grafo generador \(G\), conexo y con \(n-1\) aristas, lo que implica que \(T + e - f\) es árbol generador de \(G\).
\end{demo}

\subsection{Recorrido de árboles o grafos}
Hay dos formas de recorrer un grafo: \textbf{Breadth-First Search (BFS)} o \textbf{Depth-First Search (DFS)}.

En el BFS, se comienza por el nivel 0 (la raíz) y se visita cada vértice en un nivel antes de pasar al siguiente.

En el DFS, se comienza por la raíz y se explora cada rama lo más profundo posible antes de retroceder.

\begin{center}
	\begin{tikzpicture}[graphStyle]
	\node[basicNode] (v1) at (0,0) {1};
	\node[basicNode] (v2) [above=of v1] {2};
	\node[basicNode] (v3) [right=of v1] {3};
	\node[basicNode] (v4) [right=of v2] {4};
	\node[basicNode] (v5) [above right=of v3, yshift=-0.12cm] {5};
	\node[] (glabel) [below right=of v1, xshift=-1.75cm] {Orden de recorrido BFS};
	
	\node[basicNode] (bv1) at (5,0) {1};
	\node[basicNode] (bv2) [above=of bv1] {5};
	\node[basicNode] (bv3) [right=of bv1] {2};
	\node[basicNode] (bv4) [right=of bv2] {4};
	\node[basicNode] (bv5) [above right=of bv3, yshift=-0.12cm] {3};
	\node[] (bglabel) [below right=of bv1, xshift=-1.75cm] {Orden de recorrido DFS};
	
	\path
	(v1) edge (v2)
	(v1) edge (v3)
	(v2) edge (v3)
	(v2) edge (v4)
	(v3) edge (v5)
	(v4) edge (v5)
	
	(bv1) edge (bv2)
	(bv1) edge (bv3)
	(bv2) edge (bv3)
	(bv2) edge (bv4)
	(bv3) edge (bv5)
	(bv4) edge (bv5)
	;
	\end{tikzpicture}
\end{center}

Para ambos tipos de recorrido el algoritmo es similar, se diferencian en las estructuras que se usan para implementarlos:

\begin{algorithmic}
	\Procedure{BFS}{$\mathtt{G = (V, X)}$}
		\State $\mathtt{r \gets v \in V}$
		\State $\mathtt{to\_visit \gets Queue()}$
		\State $\mathtt{to\_visit.push(r)}$
		\State $\mathtt{r.used \gets true}$
		\While{$\mathtt{\lnot to\_visit.empty()}$}
			\State $\mathtt{i \gets to\_visit.pop()}$
			\For{$\forall~\mathtt{(i, j) \in X~/~\lnot j.used}$}
				\State $\mathtt{j.used \gets true}$
				\State $\mathtt{to\_visite.push(j)}$
			\EndFor
		\EndWhile
	\EndProcedure
\end{algorithmic}

Para implementar un \textbf{DFS} hay que utilizar una pila (\texttt{Stack}) en vez de una cola.

\subsection{Árbol generador mínimo}
\paragraph{Grafo pesado:} Grafo que tiene un costo asociado a sus aristas o vértices.

Sea \(T=(V,X)\) un árbol y \(l:X\to\reales\) una función que asigna costos a las aristas de \(T\). Se define el \textbf{costo} de \(T\) como \(l(T) = \sum_{e\in T}l(e)\).

Dado un grafo \(G =(V,X)\), un \textbf{árbol generador mínimo} de \(G\), \(AGM(G) = T	\), es un árbol generador de \(G\) de mínimo costo, es decir:
\[l(T) \leq l(T')~\forall~T'~\text{árbol generador de} G\]
\vspace*{5mm}
\begin{center}
	\begin{tikzpicture}[graphStyle]
	\node[basicNode] (v1) at (0,0) {};
	\node[basicNode] (v2) [above=of v1] {};
	\node[basicNode] (v3) [right=of v1] {};
	\node[basicNode] (v4) [right=of v2] {};
	\node[basicNode] (v5) [above right=of v3, yshift=-0.12cm] {};
	\node[] (glabel) [below right=of v1, xshift=-2cm] {Grafo pesado \(G\) y un \red{AGM(\(G\))}};
	
	\path
	(v1) edge[redEdge]  node[left]  {$1$} (v2)
	(v1) edge node[below]  {$2$} (v3)
	(v2) edge[redEdge]  node[above, xshift=0.2cm, yshift=-0.1cm] {$2$} (v3)
	(v2) edge node[above] {$5$} (v4)
	(v3) edge[redEdge]  node[below, xshift=0.2cm, yshift=0.1cm] {$2$} (v5)
	(v4) edge[redEdge] node[above, xshift=0.1cm, yshift=-0.1cm] {$1$} (v5)
	;
	\end{tikzpicture}
\end{center}

Dado un grafo pesado en las aristas \(G = (V,X)\), el problema de árbol generador mínimo consiste en encontrar un AGM de \(G\).

\subsubsection{Algoritmo de Prim}
El algoritmo de Prim es un algoritmo \textbf{goloso} que construye incrementalmente dos conjuntos, uno de vértices \(V_T\) y uno de aristas \(X_T\) que comienza vacío. En cada iteración se agrega un elemento a cada uno de estos conjuntos. Cuanto \(V_T = V\) el algoritmo termina y las aristas de \(X_T\) definen un AGM de \(G\).

En cada paso, se selecciona la arista de menor costo entre las que tiene un extremo en \(V_T\) y el otro en \(V\backslash V_T\). Esta arista es agregada a \(X_T\) y el extremo a \(V_T\).

\begin{algorithmic}
	\Procedure{Prim}{$\mathtt{G = (V, X)}$}
	\State $\mathtt{V_T \gets \{ u \}}$
	\Comment{Comenzamos con cualquier vértice de \texttt{G}}
	\State $\mathtt{X_T \gets \emptyset}$
	\State $\mathtt{i \gets 1}$
	\While{$\mathtt{i\leq n-1}$}
	\State $\mathtt{e \gets argmin\{l(e), e =(u,w),~u\in V_T \land w\in V\backslash V_T\}}$
	\State $\mathtt{X_T \gets X_T \cup \{e\}}$
	\State $\mathtt{V_T \gets V_T \cup \{w\}}$
	\State $\mathtt{i\gets i + 1}$
	\EndWhile
	\State \Return $\mathtt{T = (V_T, X_T)}$
	\EndProcedure
\end{algorithmic}

Notaremos \(T_k = (V_{k}, X_{k})\) al grafo que el algoritmo de Prim construyó al finalizar la iteración \(k\), para \(0 \leq k \leq n - 1\). \(T_0 = (V_0, X_0)\) se refiere a la inicialización antes de entrar a la primera iteración.
\begin{proposicion}\label{invariantePrim}
	Dado \(G=(V,X)\) un grafo conexo. \(T_k = (V_{k}, X_{k})\), \(0\leq k \leq n -1\), es árbol y subgrafo de árbol generador mínimo de \(G\).
\end{proposicion}
\begin{demo}
	Lo hacemos por inducción en las iteraciones del ciclo:
\end{demo}
\begin{demoPart}

	\begin{multicols}{2}
	\paragraph{Caso base (k = 0):} Antes de ingresar al ciclo, \blue{\(T_0 = (\{u\},\emptyset)\)} es árbol y subgrafo de todo AGM de \(G\).	
	\columnbreak
	\begin{center}	
		\begin{tikzpicture}[graphStyle]
		\node[basicNode, scale=0.37] (v1) at (0,0) {};
		\node[basicNode, scale=0.37] (v2) [above=of v1] {};
		\node[basicNode, scale=0.37] (v3) [right=of v1] {};
		\node[blueNode, scale=0.37] (v4) [right=of v2] {};
		\node[basicNode, scale=0.37] (v5) [above right=of v3, yshift=-0.12cm] {};
		\node[emptyDemoNode] (u) [above=of v4, yshift=-0.75cm] {\(u\)};
		\node[emptyDemoNode] (glabel) [below right=of v1, xshift=-2cm] {Paso \blue{\(T_0\)} de Prim sobre \(G\)};
		\path
		(v1) edge node[emptyDemoNode,left]  {$1$} (v2)
		(v1) edge node[emptyDemoNode,below, yshift=-0.1cm]  {$2$} (v3)
		(v2) edge node[emptyDemoNode,above, xshift=0.2cm, yshift=-0.1cm] {$2$} (v3)
		(v2) edge node[emptyDemoNode,above] {$5$} (v4)
		(v3) edge node[emptyDemoNode,below, xshift=0.1cm, yshift=-0.1cm] {$2$} (v5)
		(v4) edge node[emptyDemoNode, above, xshift=0.1cm, yshift=0.05cm] {$1$} (v5);
		\end{tikzpicture}
	\end{center}
	\end{multicols}
	\paragraph{Paso inductivo:} Consideremos \(T_k\) con \(k > 1\), nuestra hipotesis inductiva es: 
	\(\forall~k' < k,~T_k' \) es árbol y subgrafo de algún AGM \(T = (V, X_T)\) de \(G\).
	\vspace*{-0.3cm}
	\begin{multicols}{2}
	Llamemos \blue{\(w\)} al vertice agregado en la iteración \(k\) y \blue{\(e\)} a la arista. Es decir \blue{\(T_k = (V_k, X_k)\)} con \(\blue{V_k} = \red{V_{k-1}}\cup\{\blue{w}\}\) y \(\blue{X_k} = \red{X_{x-1}}\cup\{\blue{e} = (\red{u},\blue{w})\}\), \(\red{u\in V_{k-1}}\) y \(\blue{w}\notin \red{V_{k-1}}\).
	\columnbreak
		\begin{center}	
		\begin{tikzpicture}[graphStyle]
		\node[blueNode, scale=0.37] (v1) at (0,0) {};
		\node[basicNode, scale=0.37] (v2) [above=of v1] {};
		\node[redNode, scale=0.37] (v3) [right=of v1] {};
		\node[redNode, scale=0.37] (v4) [right=of v2] {};
		\node[redNode, scale=0.37] (v5) [above right=of v3, yshift=-0.12cm] {};
		\node[emptyDemoNode] (u) [below=of v3, yshift=0.65cm] {\(\red{u}\)};
		\node[emptyDemoNode] (w) [below=of v1, yshift=0.65cm] {\(\blue{w}\)};
		\node[emptyDemoNode] (glabel) [below right=of v1, xshift=-2cm] {Paso \blue{\(T_k\)} de Prim sobre \(G\)};
		\path
		(v1) edge node[emptyDemoNode,left]  {$1$} (v2)
		(v1) edge[blueEdge] node[emptyDemoNode,below, yshift=-0.1cm]  {$2$} (v3)
		(v2) edge node[emptyDemoNode,above, xshift=0.2cm, yshift=-0.1cm] {$2$} (v3)
		(v2) edge node[emptyDemoNode,above] {$5$} (v4)
		(v3) edge[redEdge] node[emptyDemoNode,below, xshift=0.1cm, yshift=-0.1cm] {$2$} (v5)
		(v4) edge[redEdge] node[emptyDemoNode, above, xshift=0.1cm, yshift=0.05cm] {$1$} (v5);
		\end{tikzpicture}
		\end{center}
\end{multicols}
\begin{itemize}	
	\item\textbf{\blue{\(T_k\)} es un árbol:} Por HI, \red{\(T_{k-1}\)} es árbol. Como \blue{\(T_k\)} tiene un vértice y un arista más que \red{\(T_{k-1}\)} y es conexo entonces \blue{\(T_k\)} es árbol.
	
	\item  \textbf{\blue{\(T_k\)} es subgrafo de un AGM:} Sea \(\violet{T = (V, X_T)}\) un AGM tal que \red{\(T_{k-1}\)} es súbgrafo \(T\) (\(T\) existe por HI):
	 \begin{itemize}
		\item Si \(\blue{e}\in \violet{X_T}\)  entonces \(\blue{T_{k}}\) también es subgrafo de \(\violet{T}\).
	
	 	\item Si \(\blue{e}\notin X_T\), veamos que podemos armar un AGM a partir de \(T\) que contenga \red{\(T_{k-1}\)} y a \(\blue{e}\):
		 		
	 	\begin{center}	
	 		\begin{tikzpicture}[graphStyle]
	 		\node[basicNode, scale=0.37] (v1) at (0,0) {};
	 		\node[basicNode, scale=0.37] (v2) [above=of v1] {};
	 		\node[redNode, scale=0.37] (v3) [right=of v1] {};
	 		\node[redNode, scale=0.37] (v4) [right=of v2] {};
	 		\node[redNode, scale=0.37] (v5) [above right=of v3, yshift=-0.12cm] {};
	 		\node[emptyDemoNode] (u) [below=of v3, yshift=0.65cm] {\(\red{u}\)};
	 		\node[emptyDemoNode] (w) [below=of v1, yshift=0.65cm] {\(\blue{w}\)};
	 		\node[emptyDemoNode] (glabel) [below right=of v1, xshift=-1.75cm] {Arbol \violet{\(T\)} tal que \(\blue{e}\notin \violet{X_T}\)};
	 		\path
	 		(v1) edge[violetEdge] node[emptyDemoNode,left]  {$1$} (v2)
	 		(v1) edge[blueEdge] node[emptyDemoNode,below, yshift=-0.1cm]  {$2$} (v3)
	 		(v2) edge[violetEdge] node[emptyDemoNode,above, xshift=0.2cm, yshift=-0.1cm] {$2$} (v3)
	 		(v2) edge node[emptyDemoNode,above] {$5$} (v4)
	 		(v3) edge[violetEdge] node[emptyDemoNode,below, xshift=0.1cm, yshift=-0.1cm] {$2$} (v5)
	 		(v4) edge[violetEdge] node[emptyDemoNode, above, xshift=0.1cm, yshift=0.05cm] {$1$} (v5);
	 		\end{tikzpicture}
	 	\end{center}
	 		 	
	 	Como \(T\) es un árbol, \(T + \blue{e}\) tiene un circuito simple \(C\) que contiene a \(\blue{e}\) (por teorema \ref{equivalenciasArbol}). Este circuito está formado por el único camino \(P_{uw}\) entre \(\red{u}\) y \(\blue{w}\) que tiene \(T\) más la arista \blue{\(e\)}.
	 	
	 	Sea \(\orange{f} \in \violet{X_T}\) la primer arista de \(P_{uw}\) tal que tiene un extremo em \(\red{T_{k-1}}\) y el otro no (existe, porque el camino empieza en \(\red{u}\) que esta dentro de \(\red{T_{k-1}}\) y termina en \(\blue{w}\) que no lo está.
	 	
	 	\begin{center}	
	 		\begin{tikzpicture}[graphStyle]
	 		\node[basicNode, scale=0.37] (v1) at (0,0) {};
	 		\node[basicNode, scale=0.37] (v2) [above=of v1] {};
	 		\node[redNode, scale=0.37] (v3) [right=of v1] {};
	 		\node[redNode, scale=0.37] (v4) [right=of v2] {};
	 		\node[redNode, scale=0.37] (v5) [above right=of v3, yshift=-0.12cm] {};
	 		\node[emptyDemoNode] (u) [below=of v3, yshift=0.65cm] {\(\red{u}\)};
	 		\node[emptyDemoNode] (w) [below=of v1, yshift=0.65cm] {\(\blue{w}\)};
	 		\path
	 		(v1) edge[violetEdge] node[emptyDemoNode,left]  {$1$} (v2)
	 		(v1) edge[blueEdge] node[emptyDemoNode,below, yshift=-0.1cm]  {$2$} (v3)
	 		(v2) edge[orangeEdge] node[emptyDemoNode,above, xshift=0.2cm, yshift=-0.1cm] {$2$} (v3)
	 		(v2) edge node[emptyDemoNode,above] {$5$} (v4)
	 		(v3) edge[violetEdge] node[emptyDemoNode,below, xshift=0.1cm, yshift=-0.1cm] {$2$} (v5)
	 		(v4) edge[violetEdge] node[emptyDemoNode, above, xshift=0.1cm, yshift=0.05cm] {$1$} (v5);
	 		\end{tikzpicture}
	 	\end{center}
	\end{itemize}
\end{itemize}
\end{demoPart}
\begin{demoPart}
	\begin{itemize}[label={}]
		\item
		\begin{itemize}[label={}]
			\item Definimos \(T' = \violet{T} + \blue{e} - \orange{f}\), veamos que es AGM:
			\begin{itemize}
				\item \(T'\) es un árbol generador de \(G\) por el teorema \ref{arbolesGeneradores}
				\item \(\blue{T_k}\) es subgrafo de \(T'\)
				\item \(T'\) es AGM de G: Como \orange{\(f\)} era una arista elegible al comienzo de la iteración \(k\), pero el algoritmo eligió \blue{e}, seguro se cumple \(l(\blue{e}) \leq l(\blue{f})\). Entonces \[l(T') = l(\violet{T}) + l(\blue{e}) - l(\blue{f}) \leq \violet{T}\]
				Entonces \(T'\) es un AGM y además \(\blue{T_k}\) es un subgrafo de \(T'\).
			\end{itemize}
		\end{itemize}
	\end{itemize}
\end{demoPart}

\begin{theorem}
	El algoritmo de Prim es correcto, es decir dado un grafo \(G\) conexo determina un árbol generador mínimo de \(G\).
\end{theorem}
\begin{demo}
	Al analizar la iteración \(n-1\), por la propocisión \ref{invariantePrim}, \(T_{n-1}\) es un árbol y subgrafo de algún \(T\) AGM de \(G\).
	
	Además, \(T_{n-1}\) es subgrafo generador de \(G\)) ya que cada iteración del algoritmo agrega un vértice distinto a \(V_T\) y, entonces, \(V_{n-1} = V\).
	
	Entonces \(T_{n-1} = T\), AGM de \(G\)
\end{demo}

\subsubsection{Algoritmo de Kruskal}
Este algoritmo ordena las aristas del grafo de forma creciente según su peso y en cada paso elige la siguiente arista que no forme circuito con las aristas ya elegidas. También es un algoritmo goloso.

\begin{algorithmic}
	\Procedure{Kurskal}{$\mathtt{G = (V, X)}$}
	\State $\mathtt{X_T \gets \emptyset}$
	\State $\mathtt{i \gets 1}$
	\While{$\mathtt{i\leq n-1}$}
	\State $\mathtt{e \gets argmin\{l(e),~e}$ \texttt{no forma circuito con las aristas de }$\mathtt{X_T\}}$
	\State $\mathtt{X_T \gets X_T \cup \{e\}}$
	\State $\mathtt{i\gets i + 1}$
	\EndWhile
	\State \Return $\mathtt{T = (V, X_T)}$
	\EndProcedure
\end{algorithmic}

Al comenzar el algoritmo, cuando todavía no se seleccionó arista alguna, cada vértice del grafo forma una componente conexa distinta (es un bosque de árboles triviales). En cada iteración, se elige una arista que tiene extremos en dos componentes conexas distintas del grafo obtenido en el paso anterior, convertindiose en nuevo bosque. El algoritmo termina cuando el bosque pasa a ser un árbol, es decir, se vuelve conexo.

La demostración de correctitud de Kruskal es similar a la de Prim: En cada iteración, el grafo que arma el algoritmo es un bosque. Al final la última iteración, como se incorporaron \(n-1\) aristas, pasa a ser árbol (sin circuitos y con \(m-1\) aristas).

En este caso, debemos demostrar que vale el invariante:

\begin{proposicion}\label{invarianteKruskal}
	Sea \(T_k\) el grafo generado en la \(k\)-ésima iteración,
	\(T_k = (V_{k}, X_{k})\), \(0\leq k \leq n -1\), es un bosque y subgrafo de árbol generador mínimo de \(G\).
\end{proposicion}

\begin{demo}
	La demostración es muy similar a la de la proposición \ref{invariantePrim}. Si \(\blue{e}=(u,w)\) es la arista incorporada en el \(k\)-ésimo lugar, la única diferencia significativa es la definición de la arista \orange{\(f\)} en el paso inductivo cuando \(\blue{e}\notin\violet{T}\) (AGM(G) tal que \(\red{T_{k-1}}\in\violet{T}\)).
	
	En este caso, debemos elegir una \orange{\(f\)} que pertenezca al circuito \(C = P_{uw} + e\) que tenga un extremo en la componente conexa a la que pertenece \(u\) y el otro fuera de esa compenente.
	
	
\end{demo}

\newpage
\section{Caminos mínimos}
Los grafos (pesados o no) son la estructura natural para modelar redes en las cuales uno quiere ir de un punto a otro de la red atravesando una secuencia de enlaces. Formalmente, sea \(G = (V, X)\) un grafo y \(l: X \to \reales\) una función de longitud/peso para las aristas de \(G\):

\begin{itemize}
  \item La \textbf{longitud} de un camino \(C\) entre dos vértices \(v\) y \(w\) es la suma de las longitudes de las aristas del camino:
        \[l(C) = \sum_{e\in C}l(e)\]
  \item Un \textbf{camino mínimo} \(C^0\) entre \(v\) y \(w\) es un caminot entre \(v\) y \(w\) tal que \[l(C^0)=\min\{l(C) | C \text{es un camino entre } v \text{ y } w\}\]
\end{itemize}

Dado un grafo \(G\), se pueden definir tres variantes de problemas sobre caminos mínimos:\
\begin{itemize}
  \item \textbf{Único origen - único destino:} Determinar un camino mínimo entre dos vértices específicos \(v\) y \(w\).
  \item \textbf{Único origen - múltiples destinos:} Determinar un camino mínimo desde un vértice específico \(v\) al resto de los vértices de \(G\).
  \item \textbf{Múltiples origenes - múltiples destinos:} Determinar una camino mínimo entre todos los vértices de \(G\).
\end{itemize}

Generalmente, los algoritmos para resolver problemas de camino mínimo se basan en que todo subcamino de un camino mínimo entre dos vértices es un camino mínimo.

\begin{proposicion}\label{prop::CaminoMinimoEstaHechoDeMinimos}
  Dado un grafo \(G = (V, X)\) con una función de peso \(l: X \to \reales\), sea \(P: v_1\dots v_k\) un camino mínimo de \(v_1\) a \(v_k\). Entonces \(\forall~1\leq i \leq j \leq k\), \(P_{v_iv_j}\) es un camino mínimo desde \(v_i\) a \(v_j\).
\end{proposicion}
\begin{demo}
  Podemos descomponer al camino \(P\) en \(P_{v_1v_i} + P_{v_iv_j} + P_{v_jv_k}\), entonces \(l(P) = l(P_{v_1v_i}) + l(P_{v_iv_j})) + l(P_{v_jv_k})\).

  Por el absurdo, asumamos que \(P_{v_iv_j}\) no es un camino mínimo desde \(v_i\) a \(v_j\). Llamemos \(P'\) a un camino entre estos dos vértices, entonces \(l(P') \leq P_{v_iv_j}\). Entonces podemos armar \(P'' = P_{v_1v_i} + P' + P_{v_jv_k}\) y
  \[l(P'') = l(P_{v_1v_i}) + l(P') + l(P_{v_jv_k}) < l(P_{v_1v_i}) + l(P_{v_iv_j})) + l(P_{v_jv_k}) = l(P)\]

  Luego \(P\) no es un camino mínimo entre \(v\) y \(w\).
\end{demo}

Dos consideraciones a tener en cuenta:
\begin{itemize}
  \item \textbf{Aristas con peso negativo:} Si el digrafo \(G\) no contiene ciclos de peso negativo o contiene alguno pero no es alcanzable desde \(v\), entonces el problema sigue estando bien definido, aunque algunos caminos puedan tener longitud negativa. Sin embargo, si \(G\) tiene algún circuito de peso negativo alcanzable desde \(v\), el concepto de camino mínimo deja de estar bien definido: Si \(w\) pertenece a un camino con ciclo de peso negativo, ningún camino de \(v\) a \(w\) puede ser mínimo porque siempre se puede obtener un camino de peso menor siguiendo ese camino pero atravesando el ciclo de peso negativo una vez más.
  \item \textbf{Circuitos:} Siempre existe un camino mínimo que no contiene circuitos (si el problema está bien definido):
        \begin{itemize}
          \item Sabemos que no puede tener un circuito negativo porque si no no está bien definido.
          \item Si tiene un circuito de peso positivo, entonces podemos sacarlo obteniendo un camino con el mismo origen y destino de menor longitud (osea que no era camino mínimo).
          \item Si tiene un circuito de peso cero, entonces sacandolo obtenemos un camino sin cirtutos del mismo peso.
        \end{itemize}
\end{itemize}

\subsection{Camino mínimo con un único origen y múltiples destinos}
Dado \(G = (V, X)\) un digrafo, \(l: X\to\reales\) una función que asigna a cada arco una cierta longitud y \(v\in V\) un vértice del digrafo, queremos calcular los caminos mínimos desde \(v\) al resto de los vértices.

\subsubsection{BFS}
En el caso de que todas las aristas tengan igual longitud, este problema se traduce en encontrar los caminos que definan las distancias (de menor cantidad de aristas). Para esto se puede adaptar el BFS (Sección \ref{algorithm:BFS}):

\begin{algorithmic}
  \Procedure{BFS}{$\mathtt{G = (V, X)}$, $v \in V$}
  \State $\mathtt{pred[w] =}$ \texttt{antecesor de w en un camino minimo desde v}
  \State $\mathtt{dist[w] =}$ \texttt{distancia desde v a w}

  \State $\mathtt{pred[v]\gets 0}$
  \State $\mathtt{dist[v]\gets 0}$
  \State $\mathtt{to\_visit \gets \{v\}}$
  \For{$\mathtt{w\in V \backslash \{ v \}}$}
  \State $\mathtt{dist[w] \gets \infty}$
  \EndFor
  \While{$\mathtt{\lnot to\_visit.empty()}$}
  \State $\mathtt{x \gets to\_visit.pop()}$
  \For{$\mathtt{w \texttt{ tal que } (x \to w)\in X~/~\land dist[w] = \infty}$}
  \State $\mathtt{pred[w] \gets x}$
  \State $\mathtt{dist[w] \gets dist[x] + 1}$
  \State $\mathtt{to\_visite.push(w)}$
  \EndFor
  \EndWhile
  \State\Return \texttt{[pred, dist]}
  \EndProcedure
\end{algorithmic}

\begin{lema}
  Dado \(G=(V,X)\) un digrafo y \(v\in V\). Sea \(\texttt{to\_visit}_k=[v_1,\dots,v_r]\) la cola de nodos que se obtiene al finalizar la iteración \(k\) del algoritmo. Se cumple que todos los nodo de la misma están a distancia \(\texttt{dist}[v_1]\) o \( \texttt{dist}[v_1]+ 1\) de \(v\), formalmente:
  \begin{itemize}
    \item \(\texttt{dist}[v_1] + 1 \geq \texttt{dist}[v_r]\) y
    \item \(\texttt{dist}[v_i] \leq \texttt{dist}[v_{i+1}]\) para todo \(i = 1,\dots r-1\)
  \end{itemize}
\end{lema}
\begin{demo}
  Lo hacemos por inducción en las iteraciones del algoritmo.
  \paragraph{Caso base:} Antes de entrar al ciclo, la cola solo tiene \(v\) y se asigna \(\texttt{dist}[v] = 0 \). Como \(v\) es el único elemento de la cola, se cumple la propiedad.


  \paragraph{Paso inductivo:} Consideremos la iteración \(k\). Llamemos \(\texttt{to\_visit}_k\) con \(k \geq 1\) al valor de la variable \texttt{to\_visit} al finalizar la iteración \(k\) del algoritmo.
\end{demo}
\begin{demoPart}
  Nuestra hipotesis inductiva es: Para todo \(k' < k\), vale que \(\texttt{to\_visit}_{k'} = [v_1^{k'},\dots,v_{r_k'}^{k'}]\) cumple que  \(\texttt{dist}[v_1] + 1 \geq \texttt{dist}[v_r]\) y
  \(\texttt{dist}[v^{k'}_i] \leq \texttt{dist}[v^{k'}_{i+1}]\) para todo \(i = 1,\dots r_{k'}-1\).

  \begin{itemize}
    \item Si \(\texttt{to\_visit}_{k-1}\) tiene un solo elemento (el que se desencola), como todos los vértices que ingresan tienen igual valor en \texttt{dist}, se cumple la propiedad.
    \item Si \(\texttt{to\_visit}_{k-1} = [u_1,u_2,\dots,u_r]\) tiene más de un elemento, sea \(w_1,\dots,w_s\) los vértices que se encolan en la iteración \(k\).
          Entonces veamos que \(\texttt{to\_visit}_k = [u_2,\dots,z,w_1,\dots,w_s]\) cumple la propiedad:
          \begin{itemize}
            \item Como \(w_s\) fue agregada en este paso del algoritmo, sabemos que \(\texttt{dist}[w_s] = \texttt{dist}[u_1] + 1\). Por hipotesis inductiva sabemos que \(\texttt{dist}[u_1] \leq \texttt{dist}[u_2]\)
                  \begin{align*}
                    \texttt{dist}[u_1] \leq \texttt{dist}[u_2] & \implies \texttt{dist}[u_1] + 1\leq \texttt{dist}[u_2] + 1 \\ & \implies \texttt{dist}[w_s] \leq \texttt{dist}[u_2] + 1
                  \end{align*}
                  Entonces la primer parte del lema se cumple.
            \item Todos los nodos encolados en este paso están a \(\texttt{dist}[u_1] + 1\) de \(v\). Osea que \(\texttt{dist}[w_i] = \texttt{dist}[u_1] + 1~\forall~i=1\dots s\), entonces tambien vale que \(\texttt{dist}[w_i] \leq \texttt{dist}[w_{i+1}]\) .

                  Por hipotesis inductiva sabemos que \(\texttt{dist}[u_i] \leq \texttt{dist}[u_{i+1}]\) para todo \(i=1\dots r\).

                  Por hipotesis inductiva sabemos \(\texttt{dist}[u_1]+1 \geq \texttt{dist}[u_r]\), luego \(\texttt{dist}[w_1] \geq \texttt{dist}[u_r]\).

                  Luego se cumple la segunda parte de la propiedad.
          \end{itemize}
  \end{itemize}
  \(\null\hfill\blacksquare\)
\end{demoPart}

\begin{coro}
  Dago \(G=(V,X)\) un digrafo y \(v,w\in V\). Si el vértice \(w\) ingresa a \texttt{to\_visit} antes que el vértice \(v\), entonces se cumple que \(dist[w]\leq dist[v]\) durante toda la corrida del algoritmo.
\end{coro}

\begin{lema}\label{lema::BFS::distMayorAd}
  Dado \(G=(V, X)\) un digrafo y \(v\in V\). Se cumple que \(\texttt{dist}[w]\geq d(v,w)\) para todo \(w\in V\) en todo momento del algoritmo.
\end{lema}

\begin{demo}
  Haremos inducción en las iteraciones de \(k\).

  \paragraph{Caso base:} Para \(k=0\) (antes de entrar al ciclo), \(\texttt{to\_visit}_0 = [v]\), \(\texttt{dist}[v] = 0\) y \(\texttt{dist}[w] = \infty\) para todo \(w\in V\) tal que \(w\neq v\).

  Como \(d(v,v) = 0\), se cumple que \(\texttt{dist}[w] \geq d(v,w)\) para todo \(w\in V\).

  \paragraph{Paso inductivo:} Nuestra hipotesis inductiva es: \(\texttt{dist}_{k-1}[w] \geq d(v,w)~\forall~w\in V\) donde \(\texttt{dist}_{k-1}\) es el valor de la variable \texttt{dist} al finalizar la iteración \(k-1\).

  Sea \(x\) el vértice que se desencolaen la iteración \(k\) y \(x_1,\dots,x_s\) los vértices que se encolan. La variable \texttt{dist} solo se modifica en las posiciones de \(x_1,\dots,x_s\). Es decir que \(\text{dist}_k[w] = \texttt{dist}_{k-1}[w]\) para todo \(w\in V / \{x_1,\dots,x_s\}\), entonces por hipotesis inductiva vale \(\texttt{dist}_k[w] \geq d(v,w)\) para \(w\in V / \{x_1,\dots,x_s\}\).

  Para las posiciones modificadas, tenemos que \[\texttt{dist}[x_i] = \texttt{dist}_{k-1}[x] + 1\]Por hipotesis inductiva sabemos que \(\texttt{dist}_{k-1}[x] \geq d(v,x)\), además como \(x_i\) fue agregado a la cola, sabemos que existe \((x, x_i)\in V\).

  Sea \(P_{i} = P_{vx} + (x,x_i)\) un camino entre \(v\) y \(x_i\). Si \(P_{vx}\) es un camino mínimo entre \(v\) y \(x\) entonces \[l(P_{i}) = l(P_{vx}) + 1 = d(v,x) + 1 \underset{HI}{\leq} \texttt{dist}_{k-1}[x] + 1 = \texttt{dist}[w_i] \]
\end{demo}
\begin{demoPart}
  Como \(d(v,x_1)\) es la longitud de los caminos mínimos entre \(v\) y \(x_i\), sabemos que \(d(v, x_i) \leq l(P_i)\), entonces \(d(v,x_1) \leq \texttt{dist}[w_i]\).
\end{demoPart}

\begin{theorem}
  Dado \(G=(V,X)\) un digrafo y \(v\in V\). El algoritmo BFS enunciado calcula \(d(v,w)\) para todo \(w\in V\)
\end{theorem}
\begin{demo}
  Por el lema \ref{lema::BFS::distMayorAd}, sabemos que \(\texttt{dist}[w]\geq d(v,w)\) para todo \(w\in V\). Supongamos ahora que para algún vértice no se cumple la igualdad.

  Sea \(x\in V\) el vértice con menor \(d(v,x)\) tal que \(\texttt{dist}[x] > d(v,x)\):
  \begin{itemize}
    \item Sabemos que \(v \neq x\) porque \texttt{dist}\([v] = d(v,v) = 0\) (este valor se asigna antes de entrar a la primera iteración).
    \item Sea \(P = v,\dots,y,x \) un camino mínimo desde \(v\) a \(x\), entonces
          \[d(v,x) = l(P) = d(v,y) + 1 \] (el subcamino de \(v\) a \(y\) es camino mínimo por la proposicion \ref{prop::CaminoMinimoEstaHechoDeMinimos}).

          Ahora, como \(x\) es el vértice de menor distancia a \(v\) tal que \(\texttt{dist}[x] > d(v,x)\) e \(y\) está a menor distancia de \(v\) que \(w\) entonces vale que \(\texttt{dist}[y] = d(v,y)\).

          Entonces, en la iteración del algoritmo que analizamos \(y\), tiene que pasar alguna de las situaciones siguientes:
          \begin{enumerate}
            \item \(x\) ya fue desencolado de \texttt{to\_visit}\label{menorQueY}
            \item \(x\) ya está en la cola\label{menorQueY2}
            \item \(x\) todavía no está en la cola \label{faltaX}
          \end{enumerate}

          Si ocurren la situacion \ref*{menorQueY} y \ref*{menorQueY2},como \(x\) fue encolado antes que \(y\), vale que \(\texttt{dist}[x] \leq \texttt{dist}[y] \), entonces:
          \[ d(v,x) = \texttt{dist}[y] + 1 \geq dist[x] + 1 > dist[x]\]
          Esto genera un absurdo.

          Si ocurre la situacion \ref*{faltaX}, \(x\) todavía no está en la cola entonces se lo agrega porque existe \((y,x) \in X\). Además, el algoritmo fijará \texttt{dist}\([x] = \texttt{dist}[y] + 1 = d(v,x)\), contradiciendo nuestra supocisión.
  \end{itemize}
\end{demo}

\subsubsection{Algoritmo de Dijkstra}
El algoritmo de Dijsktra es un algoritmo goloso que construye un árbol de caminos mínimos con raiz en \(v\). En cada iteración agrega el vértice más cercano a \(v\) de entre todos los que todavía no fueron agregados al árbol.

Este algoritmo asume que las longitudes de arco son positivas. El grafo puede ser orientado o no.

\begin{algorithmic}
  \Procedure{Dijsktra}{$G = (V, X),~v \in V$}
  \State $\mathtt{\pi[u] =}$ valor del camino mínimo desde $v$ a $u$
  \State $\mathtt{S \gets \{v\}}$
  \State $\mathtt{\pi[v] \gets 0}$\Comment{Inicializamos $\pi$ y $texttt{pred}$}
  \State $\mathtt{pred[v] \gets 0}$
  \For{$u\in V$}
  \If{$\mathtt{(v, u) \in X}$}
  \State $\mathtt{\pi[u] = l(v, u)}$
  \State $\mathtt{pred[u] = v}$
  \Else
  \State $\mathtt{\pi[u] = \infty}$
  \State $\mathtt{pred[u] = \infty}$
  \EndIf
  \EndFor

  \While{ $\mathtt{S \neq V}$ }\Comment{Calculamos los caminos mínimos}
  \State $\mathtt{w\gets arg\min\{\pi[u], u\in V\backslash S\}}$
  \State $\mathtt{S \gets S\cup\{w\}}$
  \For{$\mathtt{u \in V \backslash S~\land~(w,u)\in X}$ }
  \If{$\mathtt{\pi[u] > \pi[w] + l(w,u)}$}
  \State $\mathtt{\pi[u] \gets \pi[w] + l(w,u)}$
  \State $\mathtt{pred[u] = w}$
  \EndIf
  \EndFor
  \EndWhile
  \State\Return$[\mathtt{\pi}, pred]$
  \EndProcedure
\end{algorithmic}

\begin{lema}\label{lema::dijkstra::invariante}
  Dado un digrafo \(G=(V,X)\) con pesos positivos en los arcos, al finalizar la iteración \(k\) el algoritmo de Dijsktra determina, siguiendo hacia atrás \texttt{pred} hasta llegar a \(v\), un caminom mínimo entre el vértice \(v\) y cada vértice \(u \in \texttt{S}_k\), con longitud $\mathtt{\pi[u]}$.
\end{lema}

\begin{demo}
  Haremos inducción en la cantidad de iteraciones.
  \paragraph{Caso base (\(k = 0\)):} Antes de entrar por primera vez al ciclo, vale \(\mathtt{S}_0 =\{v\}\) y \(\mathtt{\pi}[v] = 0\). Esto es correcto ya que el camíno mínimo desde \(v\) hasta \(v\) es 0 por definición.

  \paragraph{Paso inductivo:} Consideremos la iteración \(k\) con \(k \geq 1\).

  Nuestra hipotesis inductiva es: Al terminar la iteración  \(k'\) (\(k' < k\)), \(\forall~u\in \texttt{S}_{k'}\), \(\pi[u]\) es la longitud de un camino mínimo de \(v\) a \(u\) finalizado en \(\texttt{pred}[u]\).

  Sea \(u\) el vértice agregado a \texttt{S} en la iteración \(k\). \(\texttt{S}_{k} = \texttt{S}_{k-1}\cup \{u\}\) con \(\texttt{pred}[u] = w\). Como \(w\in\texttt{S}_{k-1}\), por hipotesis inductiva, \(\pi[w]\) es el válor del camino mínimo desde \(v\) a \(w\) definido por \texttt{pred}. Llamemos \(P_{vw}\) a ese camino mínimo.

  Sea \(P = P_{vw} + (w,u)\) y \(P'\) otro camino de \(v\) a \(u\) y sea \(y\in P'\) el primer vértice que no está en \(\texttt{S}_{k-1}\) (debe existir porque \(v\in \texttt{S}_{k-1}\) y \(u\notin\texttt{S}_{k-1}\)):
  \begin{itemize}
    \item Si \(y = u\), sea \(x\) el predecesor de \(u\) (que está en \(\texttt{S}_k\)) entonces \(l(P') = \pi[x] + l(x,u)\)
          \begin{itemize}
            \item Si \(x\) entró en \texttt{S} después que \(w\), entonces el algoritmo seteó \(\pi[u] = \pi[w] + l(w,u)\) en la iteración en la que se agregó a \(w\). Luego, en la iteración  en la que agregó a \(x\) concluyó que \[\pi[u] = \pi[w] + l(w,u) \leq \pi[x] + l(x,u)\]

                  Osea que \(P\) es un camino mas corto que \(P'\).
            \item  Si \(x\) entró antes que \(w\), entonces el algoritmo seteó \(\pi[u] = \pi[x] + l(x,u)\) en la iteración en la que se agregó a \(x\). Luego, en la iteración en la que agregó a \(w\) concluyó que \[\pi[u] = \pi[x] + l(x,u) \geq \pi[w] + l(w,u)\]
                  Osea que el nuevo camino \(P\) es más corto que \(P'\) y termina quedando \(\pi[u] = \pi[w] + l(w,u)\).
          \end{itemize}
    \item Si \(y\neq u\), entonces ambos vértices son candidatos para ser elegidos en la iteración \(k\). Sin embargo, como se elgió a \(u\) vale que: \[\pi[w] + l(w,u) = \pi_{k-1}[u] \leq\pi_{k-1}[y]\]

          Sea \(x\) el predecesor de \(y\), como \(x\in S_{k-1}\) sabemos que \(\pi_{k-1}[y]\leq\pi[x] + l(x,y)\) (Hizo esta comparación cuando se agregó \(x\) a \texttt{S}\(_{k-1}\)).
  \end{itemize}
\end{demo}
\begin{demoPart}
  Ahora por hipotesis inductiva, \(\pi[x]\) es la longitud del camino mínimo desde \(v\) hasta \(x\) entonces:
  \begin{align*}
    l(P) & = \pi[w] + l(w,u) = \pi_{k-1}[u] \leq \pi_{k-1}[y] \leq \pi[x] + l(x,u) \\
         & \leq l(P'_{vx}) \leq l(P'_{vx}) + l(x,y) \leq l(P'_{vy}) \leq l(P')
  \end{align*}

  Entonces \(l(P)\leq l(P')\) para todo camino \(P'\) desde \(v\) a \(u\) y \(\pi[u]\) es la longitud de camino mínimo desde \(v\) a \(u\).\hfill\(\blacksquare\)
\end{demoPart}

\begin{theorem}
  Dado un digrafo \(G=(V,X)\) con pesos positivos en los arcos y \(v\in V\), el algorimo de Dijkstra determina el camino mínimo entre el vértice \(v\) y el resto de los vértices.
\end{theorem}

\begin{demo}
  En cada iteración un nuevo vértice es incorporado a \texttt{S} y el algortimo termina cuando \texttt{S} = \(V\).

  Por el lema \ref{lema::dijkstra::invariante}, al finalizar la última iteración del ciclo, el algoritmo de Dijkstra determina, siguiendo hacia atrás \texttt{pred} hasta llegar a \(v\), un camino mínimo entre el vértice \(v\) y cada vértice \(u\in V\) con longitud \(\pi[u]\).\hfill\(\blacksquare\)
\end{demo}

\paragraph{Complejidad del algoritmo:} Los pasos computacionales críticos son:
\begin{enumerate}
  \item\label{disjkstra::paso::agregarV} Encontrar el próximo vértice a agregat en \texttt{S}.
  \item\label{disjkstra::paso::actualizarPi} Actualizar \(\pi\).
\end{enumerate}

Cada uno de estos pasos se realiza \(n\) veces, veamos distintas implementaciones:

\begin{itemize}
  \item La forma más fácil de implementar \ref{disjkstra::paso::agregarV} es buscar secuencialmente el vértice que minimiza, esto se hace en \(O(n)\). En el paso \(2\), el vértice elegido tiene a lo sumon \(n\) vértices adyacentes y para cada uno podemos actualizar en \(O(1)\). Considerando esto, cada iteración es \(O(n)\), resultando \(O(n^2)\) elalgoritmo completo.
  \item Para mejorar, esta complejidad, en el paso \label{dijsktra::paso::agregarV}, podemos usar una cola de prioridad implementada sobre un heap de \(n\) elementos. Crear esta cola es \(O(n)\). Luego es posible borrar el elemento mínimo e insterar uno nuevo en \(O(\log{n})\). Además, si se mantiene un arreglo auxiliar apuntando a la posición acutal cada vértice en el heap, tambien es posible modificar el valor de \(\pi\) en \(O(1)\) y hacer las modificaciones necesarias al heap en \(O(log n)\).

        Considerando todas las iteraciones:
        \begin{itemize}
          \item la operación \ref{disjkstra::paso::agregarV} en \(O(log n)\).
          \item y \(d(u)\) incersiones/modificaciones en el heap para hacer actualizar \(\pi\) en \(O(\log{n})\) para cada vértice elegido \(u\). Osea que \ref{disjkstra::paso::actualizarPi} se hace en \(O(m*\log{n})\).
        \end{itemize}

        Luego el algoritmo completo resulta \(O(n\log{n} + m\log{n})\) = \(O(m\log(n))\). Notar que esta complejidad hace que el algoritmo sea más rápido si \(m\in O(n)\) (el grafo es ralo), si \(m\in O(n^2)\) (el grafo es denso), entonces la complejidad empeora.
\end{itemize}

\subsubsection{Algoritmo de Bellman-Ford}
El algorimo de Bellman-Ford, utiliza programación dinámica para calcular todos los caminos mínimos con origen en un nodo \(v\). Pero, a diferencia de Dijsktra, acepta grafos cuyos arcos pueden tener pesos negativos.

En cada iteración, el algoritmo va guardando una cota superior de la longitud de un camino mínimo desde \(v\) a \(u\). Va modificando todas esas cotas de manera iterativa hasta que llega a una iteración en la que ninguna de ellas cambia.

\begin{algorithmic}
  \Procedure{BellmanFord}{$G = (V, X),~v \in V$}
  \State $\mathtt{\pi[u] =}$ valor del camino mínimo desde $v$ a $u$
  \State
  \State $\mathtt{\pi[v] \gets 0}$\Comment{Inicializamos $\pi$ y $\texttt{pred}$}
  \State $\mathtt{i \gets 0}$
  \For{$u\in V \backslash \{v\}$}
  \State $\mathtt{\pi[u] \gets \infty}$
  \EndFor
  \State
  \While{ \texttt{haya cambioes en}$\mathtt{\pi}$ e \(i < n\) }\Comment{Calculamos los caminos mínimos}
  \State $\mathtt{i\gets i + 1}$
  \For{$\mathtt{u \in V \backslash \{v\}}$}
  \State $\mathtt{\pi[u] \gets \min(\pi[u], \min_{(w,u)\in X}\{\pi[w] + l(w,u)\})}$
  \EndFor
  \EndWhile
  \If{i == n}
  \State\Return\texttt{"Hay circiuitos de longitud negativa"}
  \Else
  \State\Return \(\pi\)
  \EndIf
  \EndProcedure
\end{algorithmic}

\begin{lema}
  Dados un digrafo \(G=(V,X\) y \(l:X\to\reales\) una funci de peso para las aristas de \(G\). Si \(G\) no tiene cirtuicos de longitud negativa, al finalizar la iteración \(k\), el algoritmo de Belman-Ford deteternuba los caminos mínimos de a lo sumo \(k\) arcos entre el vértice \(v\) y los demás vértices.
\end{lema}

\begin{demo}
  Haremos inducción en las iteraciones del algoritmo.

  \paragraph{Caso Base (\(k=0\)):} Antes de ingresar por primera vez al ciclo, el algoritmo fija \(\pi[v] = 0\) y \(\pi[u] = \infty\) para todo \(u\in V\backslash \{v\}\). Esto es correcto ya que el único camino posible con 0 arcos es de \(v\) a \(v\).

  \paragraph{Paso inductivo \(k \leq 1\):} Nuestra hipotesis inductiva es: Al terminar la iteración \(k'\) \( k' < k\), el algoritmo determin en \(\pi_{k'}\) las longitudesde los caminos mínimos desde \(v\) al resto de los vértices con a lo sumo \(k'\) arcos.

  Sea \(u\) un vértice tal que el camino mínimo desde \(v\) a \(u\) con menos cantidad de arcos contiene \(k\) arcos y \(P\) uno de estos caminos. Sea \(w\) el predecesor de \(u\) en \(P\). Por optimalidad de subcaminos, \(P_{vw}\) es un camino mínimo de \(v\) a \(w\) con \(k-1\) arcos. Por hipotesis inductiva, este caminoesta correctamente identificado al terminar la iteración \(k-1\) del algoritmo.

  Luego, en la iteración \(k\), \(\pi[u]\) recibe el valor correcto cuando se examina el arco \((w,u)\). Como este valor no puede ser mejorado (porque implicaria que \(P\) no es camino mínimo), no será modificado en las iteraciones sucesivas del algoritmo. Luego, el algoritmo calcula correctamente los caminos mínimos desde \(v\) al resto de los vértices.\hfill\(\blacksquare\)
\end{demo}


\newpage
\appendix
\section{Hoja de complejidades}
\subsection{Algoritmos sobre arrays}
\begin{center}
\begin{tabular}{|l|c|}
	\hline
	\textbf{Algoritmo} & \textbf{Complejidad} \\
	\hline
	\multicolumn{2}{|c|}{\cellcolor{blue!25}\textbf{De Búsqueda}}\\
	\hline
	Secuencial & \(O(n)\) \\
	\hline
	Binaria & \(O(\log{n})\) \\
	\hline
	\multicolumn{2}{|c|}{\cellcolor{blue!25}\textbf{De Ordenamiento}}\\
	\hline
	Bubblesort & \(O(n^2)\) \\
	\hline
	Quicksort & \(O(n^2)\) \\
	\hline
	Heapsort & \(O(n\log{n})\)* \\
	\hline
\end{tabular}
\end{center}

\paragraph{*} \(O(n\log{n})\) es la complejidad óptima para algoritmos de ordenamiento basados en comparaciones. 

\subsubsection{Algoritmos sobre grafos}

\begin{center}
	\begin{tabular}{|l|c|c|}
		\hline
		 & \multicolumn{2}{|c|}{\textbf{Complejidad}} \\
		 \hline
		\textbf{Algoritmo} & \textbf{Matriz de adyacencia} & \textbf{Lista de Adyacencia} \\
		\hline
		
		\hline
	\end{tabular}
\end{center}
\color{red}

Grafos Definiciones básicas: isomorfismos. Enumeración. Grafos eulerianos y hamiltonianos. Planaridad. Coloreo. Número cromático. Matching, conjunto independiente, recubrimiento. Recubrimiento de aristas y vértices.
	
Algoritmos en grafos y aplicaciones Representación de un grafo en la computadora: matrices de incidencia y adyacencia, listas. Algoritmos de búsqueda en grafos: A*. Arboles ordenados: códigos unívocamente descifrables. Algoritmos para detección de circuitos. Algoritmos para encontrar el camino mínimo en un grafo: Dijkstra, Ford, Dantzig. Planificación de procesos: PERT/CPM. Algoritmos heurísticos: ejemplos. Nociones de evaluación de heurísticas y de técnicas metaheurísticas. Algoritmos aproximados. Heurísticas para el problema del viajante de comercio. Algoritmos para detectar planaridad. Algoritmos para coloreo de grafos. Algoritmos para encontrar el flujo máximo en una red: Ford y Fulkerson. Matching: algoritmos para correspondencias máximas en grafos bipartitos. Otras aplicaciones.



Problemas NP-completos Problemas tratables e intratables. Problemas de decisión. P y NP. Maquinas de Turing no determinísticas. Problemas NP-completos. Relación entre P y NP. Problemas de grafos NP-completos: coloreo de grafos, grafos hamiltonianos, recubrimiento mínimo de las aristas, corte máximo, etc.
\end{document}

