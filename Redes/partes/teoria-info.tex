\part{Nível Físico}
\section{Teoría de la información}
\subsection{Información}
Sea \(E\) un suceso que puede presentarse con probabilidad \(P(E)\). Cuando \(E\) tiene lugar, decimos que hemos recibido \[I(E)=\log\frac
{1}{P(E)}\] unidades de información.

Si introducimos el logaritmo de base 2, la unidad se denomina \textit{bit}. Notemos, también, que si \(P(E) = \frac{1}{2}\), \(I(E) = 1\) bit. Es decir, un bit es la cantidad de información obtenida al especificar una de dos posbiles alternativas igualmente probables.

\subsubsection{Fuentes de Memoria Nula}
Son fuentes de información que emiten una secuencia de símbolos pertenecientes a un alfabeto finito y fijo \(S=\{s_1,\dots,s_n\}\) de manera estadísticamente independientes. Estas fuentes pueden describirse mediante el alfabeto \(S\) y las probabilidades con que los simbolos se presentan: \(P(s_1), \dots, P(s_n)\).

La información que aporta cada símbolo de la fuente \[I(s_i) = \log_2\frac{1}{P(s_i)}\text{ bits}\]
\subsubsection{Entropía}
Dada una fuente de memoria nula \(S\) con alfabeto \({s_1,\dots,s_n}\), la entropía \(H(S)\) es la cantidad media de información por símbolo de la fuente, es decir:
\[H(S) = \sum_{i=1}^n P(s_i)I(s_i) = \sum_{i=1}^n P(s_i)\log_2\frac{1}{P(s_i)} = -\sum_{i=1}^n P(s_i)\log_2 P(s_i)\text{ bits} \]

Podemos interpretar la entropía como el \textbf{valor medio ponderado de la cantidad de información} del conjunto de mensajes posibles, como una medida de la \textbf{incertidumbre probmedio (grado de incerteza)}acerca de una variable aleatoria o la \textbf{cantidad de información} obtenida al observar la aparición de cada nuevo símbolo.

\paragraph{Propiedades:}
\begin{itemize}
  \item La entropía es no negativa y se anula si y solo si la probabilidad de uno de sus símbolos es 1 y la del resto es 0.
  \item La entropía máxima (\textbf{mayor incertidubme del mensaje}) se logra cuando todos los símbolos que puede ser emitidos por la fuente son equiprobables.
    \item Si hay \(n\) símbolos equiprobables \(P(s)=\frac{1}{n}\) se cumple:
    \[
      H(S) = -\sum_s P(S)\log_2 P(S) = -n(\frac{1}{n}\log_2\frac{1}{n}) = -(\log_2 1 - log_2 n) = \log_2 n
    \]
\end{itemize}

\subsubsection{Extensión de memoria nula}
Si tenemos una fuente de memoria nula \(S\), con un alfabeto \(\{s_1,\dots,s_q\}\), podemos agrupar las salidas en paquetes de \(n\) símbolos. Tendremos, pues, \(qn\) secuencias de salidas distintas. 

Formalmente, sea \(S\) una fuente de información de memoria nula con un alfabeto \(\{s_1,\dots,s_q\}\). Sea \(P_i\) la probabilidad correspondiente a \(s_i\). Entonces, la extensión de orden \(n\) de \(S\) \((S^n)\) es una fuente de memoria nula de \(qn\) símbolos \(\{\sigma_1,\dots,\sigma_{qn}\}\) donde cada \(\sigma_i\) corresponde a una secuencia de simbolos de \(S\) de longitud \(n\). La probabilidad de \(\sigma_i\), \(P_{\sigma_i}\) es la probabilidad de la secuencia correspondiente. Es decir, si \(\sigma = s_{i_1}\dots s_{i_k}\) entonces \(P(\sigma) = P_{i_1}\dots P_{i_k}\).

Puesto que un símbolo de \(S^n\) corresponde a \(n\) símbolos de \(S\), es de suponer que la entropía por símbolo de \(S^n\) sea \(n\) veces mayor que la de \(S\), osea:
\[H(S^n) = nH(S)\]

\subsection{Codificación}
Sea \(S = \{ s_1,\dots, s_q\} \) el conjunto de símbolos de un alfabeto dado. Se define un \textbf{código} como la correspondencia de todas las secuencias posibles de símbolos de \(S\) a secuencias de símbolos de algún otro alfabeto \(X =\{x_1,\dots x_r\}\). \(S\) recibe el nombre de \textbf{alfabeto fuente} y \(X\) el de \textbf{alfabeto código}.

Cuando codificamos un alfabeto fuente, buscamos lograr una representación eficiente de la información mediante la eliminación de la redundancia.q

\paragraph{Código bloque:} Es aquel que asigna cada uno de los símbolos del alfabeto fuente \(S\) a una secuencia fija de símbolos del alfabeto código \(X\). Esas secuencias fijas (secuencias de \(x_i\)) reciben el nombre de palabras
código. Denominaremos \(X_i\), a la palabra código que corresponde al símbolo \(s_i\).

\paragraph{Código no singular:} Es un código en el que todas sus palabras son distintas.

\paragraph{Extensión de orden n:} La extensión \(C^n\) de orden \(n\) de un código bloque \(C:S\to X^*\), es el código bloque que hace corresponder las secuencias de símbolos de \(S\) con las secuencias de las palabras código formadas por \(C(s_i)\). Es decir: Si \(s_i\dots s_k\in S^*\), entonces \(C^n(s_i\dots s_k) = C(s_i)\dots C(s_k)\).

\paragraph{Código univócamente decodificable:} Es aquel en el cual ninguna tira de símbolos del alfabeto código admite más de una única decodificación. Dicho de otra forma, un código bloque se dice univocamente decodificable si, y solamente si, su extensión de orden \(n\) es no singular para, cualquier valor finito de \(n\).

\paragraph{Código instantaneo:} Un código unívocamente decodificable se denomina instantáneo cuando es posible decodificar las palabras de una secuencia sin precisar el conocimiento de los símbolos que las suceden.

\paragraph{Préfijo de una palabra:} Sea \(X=x_1\dots x_m\) una palabra de un código. Se denomina prefijo de esta palabra a la secuencia de símbolos \(x_1\dots x_j\), donde \(j \leq m\).

La condición necesaria y suficiente para que un código sea instantáneo es que ninguna palabra del código coincida con el prefijo de otra.

\paragraph{Inecuación de Kraft:} Dado un alfabeto \(S = \{s_1,\dots,s_n\}\) y un alfabeto de código \(X=\{x_1,\dots,x_m\}\), es condición necesaria y suficiente, para exista un código instantáneo con palabras de longitud \(l_1,\dots,l_{n}\), que se cumpla la siguiente inecuación:
\[\sum_{i=1}^n |X|^{-l_1}\leq 1\]

\subsubsection{Codificación óptima}
Buscamos codificar un alfabeto \(S\) de tal forma que máximizar la relación entre la entropía \(H(S)\) y la longitud media del código \(L\).  Sea \(l_i\) la longitud de la palabra que codifica al símbolo \(s_i\) de la fuente, \(p_i\) la probabilidad de aparicion de \(s_i\) y \(r\) la cantidad de símbolos diferentes en el alfabeto del código entonces:

\begin{itemize}
  \item \(L = \sum p_iL_i \) es la longitud medía del código.
  \item \(\log r\) es la cantidad promedio máxima de información de un símbolo del código.
  \item \(h = \frac{H(S)}{L\log r}\) es la eficencia del código.
\end{itemize}

La máxima eficencia se logra cuando \(h = 1\). En general, esto sucede cuando se asigna las palabras de código más cortas a los símbolos de fuente más probables. También se puede deducir: \[1 \geq h \geq \frac{H(S)}{L\log r}\Rightarrow L\log r \geq H(S)\]

\paragraph{Primer teorema de Shannon (Teorema de la codificación sin ruido):} Sea \(S\) una fuente de memoria no nula, y \(S^n\) la extensión de orden \(n\) de \(S\).  Sea \(C:S^n\to X^n\) un código y \(L_n\) la longitud media de los códigos correspondientes a los símbolos de \(S^n\):
\[
    L_n = \sum_{\sigma\in S^n} C(\sigma)P(\sigma)
\]
Entones vale:
\[
    H(S) \leq \frac{L_n}{n} \leq H(S) + \frac{1}{n}
\]

Esto nos dice que el número medio de símbolos de \(C\) correspendientes a un símbolo de la fuente puede hacerse tan pequeño (pero no inferior) como la entropía de la fuente. El precio que se paga por la disminución de \(L_n\) es un aumento en la complejidad de la codificación debido al gran número de símbolos de la fuente que hay que manejar.

En particular:
\[
  H(S) = \sum_{\sigma\in S^n} C(\sigma)P(\sigma)  
\]

\paragraph{Codificador óptimo:} Es un codificador que usa la menor cantidad posible de bits para codificar un mensaje, es decir: Un codificador se dice óptimo si no existe ningún código para la misma fuente con menor longitud media.

Sea \(s_i\in S\), entonces la cantidad de bits necestarios para representarlo en un codificador óptimo es \(\lceil\frac{1}{P(s)}\rceil\) y la entropía de \[H(X) = \sum_{s\in S} P(s)\log_2\left(\frac{1}{P(s)}\right)\]

\paragraph{Codificación de Huffman:} Es una forma de definir códigos óptimos asumiendo que se conoce la probabilidad de ocurrencia de los símbolos, que la codificación es símbolo por símbolo y la probabilidad de ocurrencia de cada símbolo es independiente.

Dado un mensaje \(M\):
\begin{enumerate}
  \item Se extrae del mismo la frecuencia de cada símbolo.
  \item Se ordenan los símbolos en arbol dependiendo de la frecuencia de cada uno. Mientras más cerca de la raíz, más frecuente es el símbolo.
  \item El código de un símbolo será entonces el camino de la raíz al nodo donde está ubicado (utilizando
  ceros cuando se toma la rama izquierda y 1 cuando se toma la rama derecha).
\end{enumerate} 

Así, los símbolos más frecuentes tendrán los códigos más cortos.



