\section{Autómatas finitos}
\subsection{Autómatas finitos deterministicos (AFD)}
Un autómata finito determinista es una 5-tupla \(\mathcal{M}=\langle Q,\Sigma,\delta,q_0,F\rangle\) donde:
\begin{itemize}
  \item \(Q\) es un conjunto finito de estados.
  \item \(\Sigma\) es un conjunto finito de símbolos de entrada.
  \item \(\delta:Q\times\Sigma\to Q\) es una función de transición.
  \item \(q_0\in Q\) es el estado inicial.
  \item \(F\subseteq Q\) es el conjunto de estados finales.
\end{itemize}

\paragraph{Función de transición generalizada \(\hat{\delta}\):} La función de transición \(\delta\) está definida para que tome como parámetro un único símbolo de \(
\Sigma\). Se puede extender para que tome como parámetro una cadena de símbolos de \(\Sigma\), es decir \(\hat{\delta} : Q\times\Sigma^*\to Q\):
\begin{itemize}
  \item \(\hat{\delta}(q,\lambda)=q\)
  \item \(\hat{\delta}(q,\beta a)=\delta(\hat{\delta}(q,\beta),a)\) con \(\beta\in\Sigma^*\) y \(a\in\Sigma\)
\end{itemize}

\paragraph{Cadena aceptada por un AFD:} Una cadena \(\beta\in\Sigma^*\) es aceptada por un AFD \(\mathcal{M} = \langle Q, \Sigma, \delta, q_0, F\rangle\) si y solo si \(\hat{\delta}(q_0,\beta)\in F\).

\paragraph{Lenguaje aceptado por un AFD:} El lenguaje aceptado por un AFD \(\mathcal{M} = \langle Q, \Sigma, \delta, q_0, F\rangle\) es el conjunto de todas las cadenas \(\beta\in\Sigma^*\) que son aceptadas por \(\mathcal{M}\):
\[ L(\mathcal{M}) = \{ \beta\in\Sigma^*:~\hat{\delta}(q_0,\beta)\in F\}\]

\subsection{Autómatas finitos no deterministas (AFND)}
Un autómata finito no determinista es una 5-tupla \(\mathcal{M}=\langle Q,\Sigma,\delta,q_0,F\rangle\) donde:
\begin{itemize}
  \item \(Q\) es un conjunto finito de estados.
  \item \(\Sigma\) es un conjunto finito de símbolos de entrada.
  \item \(\delta:Q\times\Sigma\to \red{\mathcal{P}(Q)}\) es una función de transición.

        A diferencia de los AFD, la función \(\delta\) devuelve un conjunto de estados en lugar de un solo estado.
  \item \(q_0\in Q\) es el estado inicial.
  \item \(F\subseteq Q\) es el conjunto de estados finales.
\end{itemize}

\paragraph{Función de transición generalizada \(\hat{\delta}\):} Primero vamos a definir \(\delta_P : \mathcal{P}(Q)\times\Sigma\to\mathcal{P}(Q)\) de la siguiente manera:
\[ \delta_P(P,a) = \underset{p\in P}{\bigcup}\delta(p,a)\]

La función \(\hat{\delta} : Q\times\Sigma^*\to \mathcal{P}(Q)\) se define de manera recursiva como:
\begin{itemize}
  \item \(\hat{\delta}(q,\lambda)=\{q\}\)
  \item \(\hat{\delta}(q,\beta a)= \{ p:~\exists r\in\hat{\delta}(q,\beta)\) tal que \(p \in \delta(r, a)\} = \delta_P(\hat{\delta}(q, \beta), a)\) con \(\beta\in\Sigma^*\) y \(a\in\Sigma\)
\end{itemize}

Para generalizar a un más podemos definir \(\hat{\delta}_P : \mathcal{P}(Q)\times\Sigma^*\to\mathcal{P}(Q)\) de la siguiente manera:
\[ \hat{\delta}_P(P,\beta) = \underset{q\in P}{\bigcup}\hat{\delta}(q,\beta)\]

\paragraph{Cadena aceptada por un AFND:} Una cadena \(\beta\in\Sigma^*\) es aceptada por un AFND \(\mathcal{M} = \langle Q, \Sigma, \delta, q_0, F\rangle\) si y solo si \(\hat{\delta}(q_0,\beta)\cap F \neq \emptyset\). Es decir, si alguno de los estados alcanzados por \(\hat{\delta}(q_0,\beta)\) es un estado final.

\paragraph{Lenguaje aceptado por un AFND:} El lenguaje aceptado por un AFND \(\mathcal{M} = \langle Q, \Sigma, \delta, q_0, F\rangle\) es el conjunto de todas las cadenas \(\beta\in\Sigma^*\) que son aceptadas por \(\mathcal{M}\):
\[ L(\mathcal{M}) = \{ \beta\in\Sigma^*:~\hat{\delta}(q_0,\beta)\cap F \neq \emptyset\}\]

\subsubsection{Equivalencia entre AFD y AFND}
Es trivial ver que para todo AFD existe un AFND que acepte el mismo lenguaje.

\begin{teorema}
  Dado una AFND \(\mathcal{M} = \langle Q, \Sigma, \delta, q_0, F\rangle\), existe un AFD \(\mathcal{M}' = \langle Q', \Sigma, \delta', q_0', F'\rangle\) tal que \(L(\mathcal{M}) = L(\mathcal{M}')\).
\end{teorema}
Vamos a demostrar este teorema construyendo una AFD \(\mathcal{M}'\) a partir de \(\mathcal{M}\). Una vez constuido deberemos demostrar que \(\mathcal{M}'\) acepta el mismo lenguaje que \(\mathcal{M}\).

\paragraph{Construcción de \(\mathcal{M}'\):}
\begin{itemize}
  \item \(Q'\) será el conjunto de partes \(\mathcal{P}(Q)\). Vamos a denotar cada estado \(s\in Q'\) con etiquetas del estilo \([q_1,\dots, q_k]\) donde \(q_1,\dots,q_k\in Q\). Entonces:
        \[ Q' = \mathcal{P}(Q)\]
  \item \(\delta'([q_1,\dots, q_k],a) = [p_1, \dots, p_m] \iff \delta_P(\{q_1,\dots,q_k\},a) = \{p_1,\dots,p_m\}\)
  \item \(q_0' = [q_0]\)
  \item \(F' = \{ [q_1,\dots, q_n]\in Q' : \{q_1,\dots,q_n\}\cap F \neq \emptyset\}\)
\end{itemize}

\paragraph{Equivalencia entre funciones de transición:} Antes de demostrar que ambos automátas aceptan el mismo lenguaje, vamos a demostrar que las funciones de transición generalizadas de ambos automátas son equivalentes cuando las llamamos con el estado inicial como primer parámetro. Es decir, queremos ver que \(\hat{\delta}'(q_0',\beta) = [p_1,\dots,p_k] \iff \hat{\delta}(q_0,\beta) = \{p_1,\dots, p_k\}\).

Lo vamos a hacer por inducción. Recordemos que \(q_0' = [q_0]\):
\begin{itemize}
  \item Caso base: \(\beta = \lambda\):
        \begin{itemize}
          \item \(\hat{\delta}'([q_0],\lambda) = [q_0]\) por definición de \(\hat{\delta}'\).
          \item \(\hat{\delta}(q_0,\lambda) = \{q_0\}\) por definición de \(\hat{\delta}\).
        \end{itemize}
        Luego \(\hat{\delta}'([q_0],\lambda) = [q_0] \iff \hat{\delta}(q_0,\lambda) = \{q_0\}\).

  \item Caso inductivo: \(\beta \implies \beta a\):

        Nuestra hipotesis inductiva es \(\hat{\delta}'(q_0',\beta) = [r_1,\dots,r_m] \iff \hat{\delta}(q_0,\beta) = \{r_1,\dots, r_m\}\).

        Queremos ver que \(\hat{\delta}'(q_0',\beta a) = [p_1,\dots,p_k] \iff \hat{\delta}(q_0,\beta a) = \{p_1,\dots, p_k\}\)
        \begin{align*}
          \hat{\delta}'(q_0',\beta a) = [p_1,\dots,p_k]  \iffa{def.} & \delta'(\hat{\delta}'(q_0',\beta),a) = [p_1,\dots,p_k]                                                            \\
          \iffa{def.}                                                & \red{\exists [r_1,\dots,r_m]\in Q' \text{ tal que } \delta'(q_0',\beta) = [r_1,\dots,r_m]}                        \\
                                                                     & \land \blue{\delta'([r_1,\dots,r_m],a) = [p_1,\dots,p_k]}                                                         \\
          \iffab{\red{H.I}}{\blue{constr.}}                          & \red{\exists \{   r_1,\dots,r_m\}\in Q \text{ tal que } \hat{\delta}(q_0,\beta) = \{r_1,\dots,r_m\}}              \\
                                                                     & \land \blue{\delta_P(\{r_1,\dots,r_m\},a) = \{p_1,\dots,p_k\}}                                                    \\
          \iffa{def.}                                                & \delta_P(\hat{\delta}(q_0,\beta),a) = \{p_1,\dots,p_k\} \iffa{def.} \hat{\delta}(q_0,\beta a) = \{p_1,\dots,p_k\}
        \end{align*}
\end{itemize}

\paragraph{Demostración de la equivalencia:} Ahora que hemos demostrado que las funciones de transición generalizadas de ambos automátas son equivalentes, vamos a demostrar que ambos automátas aceptan el mismo lenguaje:

\begin{align*}
  \beta \in \mathcal{L}(\mathcal{M})  \iffa{def.} & \red{\hat\delta(q_0, \beta) = \{q_1,\dots, q_n\}}\land\blue{\{q_1,\dots,q_n\}\cap F \neq \emptyset} \\
  \iffab{\red{equiv.}}{\blue{constr.}}            & \red{\hat\delta(q_0', \beta) = [q_1,\dots, q_n]}\land\blue{[q_1,\dots,q_n]\in F'}                   \\
  \iffa{def.}                                     & x\in\mathcal{L}(M')
\end{align*}

\subsection{Autómatas finitos no deterministico con transiciones  \texorpdfstring{\(\lambda\)}{lambda}}
\label{sec:afd-lambda}
Un autómata finito no determinista con transiciones \(\lambda\) es un autómata finito no determinista que tiene transiciones \(\lambda\). Estas transacciones nos permiten ir de un estado a otro sin consumir ningún símbolo de entrada.

Los definimos con una 5-upla \((Q,\Sigma,\delta,q_0,F)\) donde:
\begin{itemize}
  \item \(Q\) es un conjunto finito de estados.
  \item \(\Sigma\) es un conjunto finito de símbolos de entrada.
  \item \red{\(\delta : Q \times (\Sigma \cup \{\lambda\}) \to \mathcal{P}(Q)\)} es una función de transición.
  \item \(q_0 \in Q\) es el estado inicial.
  \item \(F \subseteq Q\) es el conjunto de estados finales.
\end{itemize}

\paragraph{Clausura \(\lambda\) de un estado \(q\):} Se denota \(Cl_\lambda(q)\) es el conjunto de estados que se pueden alcanzar desde \(q\) siguiendo solo transiciones \(\lambda\). Es decir, \[Cl_\lambda(q) = \delta(q,\lambda)\]

Además \(q\in Cl_\lambda(q)\).

\paragraph{Clausura \(\lambda\) de un conjunto de estados \(P\):} \[Cl_{P\lambda}(P) = \bigcup_{p\in P} Cl_\lambda(p)\]
\paragraph{Generalización de la función de transición:}
Podemos extender \(\delta\) a conjunto de estados:
\begin{align*}
   & \delta_P: \mathcal{P}(Q) \times (\Sigma \cup \{\lambda\}) \to \mathcal{P}(Q) \\
   & \delta_P(P,a) = \underset{p\in P}{\bigcup} \delta(p,a)
\end{align*}

Entonces podemos definir:

\begin{align*}
   & \hat\delta: Q \times \Sigma^* \to \mathcal{P}(Q)                                                                                                                                                             \\
   & \hat\delta(q_0,\lambda) = Cl_\lambda(q_0)                                                                                                                                                                    \\
   & \hat\delta(q_0, \beta a) = Cl_{P\lambda}\left(\delta_P(\hat\delta(q_0,\beta), a)\right) = Cl_{P\lambda}\left(\left\{ p: \exists q\in \hat\delta(q_0,\beta) \text{ tal que } p\in \delta(q,a) \right\}\right)
\end{align*}

Tambien extendemos \(\hat\delta\) a conjuntos de estados:
\begin{align*}
   & \hat\delta_P: \mathcal{P}(Q) \times \Sigma^* \to \mathcal{P}(Q)  \\
   & \hat\delta_P(P,\beta a) = \bigcup_{p\in P} \hat\delta(p,\beta a)
\end{align*}

\paragraph{Cadena aceptada por un AFND-\(\lambda\):} Una cadena \(\beta\) es aceptada por un AFND-\(\lambda\) \(M =\langle Q,\Sigma,\delta,q_0,F\rangle\) si y solo si \(\hat\delta(q_0,\beta)\cap F \neq \emptyset\).

\paragraph{Lenguaje aceptado por un AFND-\(\lambda\):} El lenguaje aceptado por un AFND-\(\lambda\) \(M =\langle Q,\Sigma,\delta,q_0,F\rangle\) es el conjunto de todas las cadenas aceptadas por \(M\):
\[\mathcal{L}(M) = \{\beta \in \Sigma^* : \hat\delta(q_0,\beta)\cap F \neq \emptyset\}\]

\subsubsection{Equivalencia entre AFND y AFND-\texorpdfstring{\(\lambda\)}{lambda}}
\label{sec:afd-afd-lambda}
Dado un AFND-\(\lambda\) \(M =\langle Q,\Sigma,\delta,q_0,F\rangle\) podemos construir un AFND \(M' =\langle Q,\Sigma,\delta',q_0,F'\rangle\) tal que \(M\) acepte el mismo lenguaje que \(M'\).

\paragraph{Construcción de M':} Notemos que ambos autómatas tiene el mismo conjunto de estados \(Q\), el mismo conjunto de símbolos de entrada \(\Sigma\) y el mismo estado inicial \(q_0\). Por lo que solo debemos definir \(\delta'\) y \(F'\).
\begin{itemize}
  \item \(\delta'(q,a) = \hat\delta(q, a) = Cl_{P\lambda}\left(\delta_P(\hat\delta(q,\lambda), a)\right)\)
  \item \(
        F' = \begin{cases}
          F\cup\{q_0\} & \text{si } Cl_{\lambda}(q_0)\cap F \neq \emptyset \\
          F            & \text{si no}
        \end{cases}\)
\end{itemize}

\paragraph{Equivalencia de funciones de transición generalizada:} Vamos a demostrar por inducción que \(\hat\delta'(q_0,\beta) = \hat\delta(q_0,\beta)\) para todo \(|\beta|\geq 1\):
\begin{itemize}
  \item Caso base: \(|\beta|=1\). Sea \(\beta = a\), entonces \(\hat\delta'(q_0,\beta) = \hat\delta'(q_0, a) = \hat\delta(q_0, a)\) por como definimos \(\delta'\).
  \item Caso inductivo: Supongamos que \(\hat\delta'(q_0,\beta) = \hat\delta(q_0,\beta)\) para todo \(|\beta|\leq n\). Sea \(\omega = \beta a\). Entonces:
        \begin{align*}
          \hat\delta'(q_0,\omega) & = \hat\delta'(q_0,\beta a) \underset{\text{def.}}{=} \delta'_P(\hat\delta'(q_0,\beta), a) \underset{\text{H.I}}{=} \delta'_P  (\hat\delta(q_0,\beta), a)\hspace*{1cm}(1) \\
        \end{align*}
        Por otro lado, dado \(P \subseteq Q\) tenemos que:
        \begin{align*}
          \delta_P'(P, a) & \underset{\text{def.}}{=} \bigcup_{p\in P} \delta'(p, a) \underset{\text{constr.}}{=} \bigcup_{p\in P} \hat\delta(p, a)\underset{\text{def}}{=} \hat\delta_P(P,a)
        \end{align*}
        Entonces, remplazando el último término en (1), nos queda:
        \begin{align*}
          \delta'_P(\hat\delta(q_0,\beta), a) & = \hat\delta_P(\hat\delta(q_0,\beta), a) \underset{\text{def.}}{=} \hat\delta(q_0,\beta a) \underset{\text{def.}}{=} \hat\delta(q_0,\omega)
        \end{align*}
\end{itemize}

\paragraph{Demostración de equivalencia:} Veamos ahora que \(M\) acepta el mismo lenguaje que \(M'\), vamos a separar la demostración en dos casos: \(\beta=\lambda\) y \(\beta \neq \lambda\).
\begin{itemize}
  \item \(\beta=\lambda\)
        \begin{itemize}
          \item \(\lambda\in\mathcal{L}(M) \implies \lambda\in\mathcal{L}(M')\)
                \begin{align*}
                  \lambda\in \mathcal{L}( M)  \iffa{def.} & \hat\delta(q_0,\lambda)\cap F \neq \emptyset     \\ \iffa{def.} &Cl_{\lambda}(q_0)\cap F \neq \emptyset \\
                  \thena{constr.}                         & q_0\in F' \iffa{def.} \lambda\in \mathcal{L}(M')
                \end{align*}

          \item \(\lambda\in\mathcal{L}(M') \implies \lambda\in\mathcal{L}(M)\).
                \begin{align*}
                  \lambda\in \mathcal{L}(M')  \thena{def.} & q_0\in F'                                                                                \\
                  \thena{constr.}                          & \red{q_0\in F} \lor \blue{Cl_\lambda(q_0)\cap F \neq\emptyset}                           \\
                  \thena{\red{def. F}}                     & \red{Cl_\lambda(q_0)\cap F \neq\emptyset}\lor \blue{Cl_\lambda(q_0)\cap F \neq\emptyset} \\
                  \thena{def.}                             & Cl_\lambda(q_0)\cap F \neq\emptyset                                                      \\
                  \iffa{def.}                              & \lambda\in\mathcal{L}(M)                                                                 \\
                \end{align*}
        \end{itemize}
  \item \(\beta \neq \lambda\)
        \begin{itemize}
          \item \(\beta\in\mathcal{L}(M) \implies \beta\in\mathcal{L}(M')\)
                \begin{align*}
                  \beta\in \mathcal{L}(M)  \thena{def.} & \hat\delta(q_0,\beta)\cap F \neq \emptyset                                                             \\
                  \thena{equiv. tran.}                  & \hat\delta'(q_0,\beta)\cap F \neq\emptyset                                                             \\
                  \thena{constr. M'}                    & \hat\delta'(q_0,\beta)\cap F' \neq \emptyset \underset{\text{def.}}{\implies} \beta\in \mathcal{L}(M')
                \end{align*}
          \item \(\beta\in\mathcal{L}(M') \implies \beta\in\mathcal{L}(M)\)
                \begin{align*}
                  \beta\in \mathcal{L}(M') \thena{def.}  & \hat\delta'(q_0,\beta)\cap F' \neq \emptyset                                                          \\
                  \thena{equiv. trans}                   & \hat\delta(q_0,\beta)\cap F \neq \emptyset\lor \hat\delta(q_0,\beta)\cap(F\cup\{q_0\}) \neq \emptyset
                  \hat\delta(q_0,\beta) \cap
                  \\
                  \underset{\text{constr.} M'}{\implies} & \hat\delta(q_0,\beta)\cap F \neq \emptyset
                \end{align*}
                Si vale la primera parte de la última expresión \(\delta(q_0,\beta)\cap F \neq \emptyset\) entonces \(\beta\in \mathcal{L}(M)\) por definición.

                Veamos que pasa si vale \(\hat\delta(q_0,\beta)\cap(F\cup\{q_0\}) \neq \emptyset\), es decir hay un camino de  transiciones \(\lambda\) desde \(q_0\) hasta algún estado estado \(q'\in F\):
                \begin{align*}
                  \hat\delta(q_0,\beta)\cap(F\cup\{q_0\}) \neq \emptyset \implies \hat\delta(q_0,\beta)\cap F \neq \emptyset \lor \hat\delta(q_0,\beta)\cap\{q_0\} \neq \emptyset
                \end{align*}

                La primer parte es lo mismo que arriba, analizemos la segunda:
                \begin{align*}
                  \hat\delta(q_0,\beta)\cap\{q_0\} \neq \emptyset \implies Cl_\lambda(q_0)\cap F \neq \emptyset \implies \lambda\in \mathcal{L}(M)
                \end{align*}
        \end{itemize}
\end{itemize}
Queda demostrada la equivalencia de lenguajes.
