\documentclass[10pt,a4paper]{article}

\usepackage{algpseudocode}
\usepackage{algorithmicx}
\usepackage{amsfonts}
\usepackage{amsmath}
\usepackage{amssymb}
\usepackage[spanish]{babel}
\usepackage{dsfont}
\usepackage{enumitem}
\usepackage{fancyhdr}
\usepackage{geometry}
\usepackage{graphicx}
\usepackage[hidelinks]{hyperref}
\usepackage{ifthen}
\usepackage[utf8]{inputenc}
\usepackage{multicol}
\usepackage{titling}
\usepackage{xcolor}
\usepackage{wrapfig}


%%%% CONFIGURACIONES %%%%

%% La coma de los reales es un punto
\decimalpoint

%%% Tamaño de pagina
\geometry{
	includeheadfoot,
	margin=2.54cm
}

%\stul{0.1cm}{0.2ex}

%% HEADER Y FOOTER
\pagestyle{fancy}

\fancyhf{}

\fancyhead[LO]{Sección \rightmark} % \thesection\ 
\fancyhead[RO]{\small{\thetitle}}
\fancyfoot[CO]{\thepage}
\renewcommand{\headrulewidth}{0.5pt}
\renewcommand{\footrulewidth}{0.5pt}
\setlength{\headsep}{1cm}
\setlength{\headheight}{13.07225pt}

\renewcommand{\baselinestretch}{1.2}  % line spacing

%% Links en indice 
\hypersetup{
	linktoc=all,     %set to all if you want both sections and subsections linked
	linkcolor=blue,  %choose some color if you want links to stand out
}

\tikzstyle{demoBox} = [
draw=gray, very thick,
rectangle split, rectangle split parts=2, rounded corners, inner sep=2.5mm, inner ysep=2mm,
rectangle split part fill = {blue!40, blue!5}
]

\NewEnviron{definicion}[1]{%
\begin{center}
	\begin{tikzpicture}
	\node [demoBox](box){%
		\textbf{\scriptsize DEFINICIÓN: #1}
		\nodepart{two}
		\begin{minipage}{\textwidth}
		\BODY
		\end{minipage}
	};
	\end{tikzpicture}
\end{center}
}
\title{Ingenieria I - Resumenes papers 2c2019}
\author{Gianfranco Zamboni}
\begin{document}
	
\maketitle
\tableofcontents
\newpage
\part{\red{Introducción}}
\section{Temas}
Los temas incluidos en este resumen incluyen:
\begin{itemize}
	\item Del libro \textbf{Principles of Model Checking} por \textit{Katoen, Baier y Larsen}:
	\begin{itemize}
		\item \textbf{Partial Order Reduction:} \textit{(Sección 8.1 y 8.2)} Presentan un método para reducir el tamaño (en cantidad de estados) de la composición de dos  sistemas de transición y como hacer model checking sobre el sistema reducido.
		\item \textbf{Timed Automata:} \textit{(Sección 9.1 y 9.2)} Extienden la definición de los sistemas de transiciones para trabar con requerimientos temporales continuos y redefinen los operadores de LTL para poder realizar model checking sobre los mismos.
		\item \textbf{Markov Chains:} \textit{(Sección 10.1 y 10.2)} 
		\item \red{\textbf{K-Inudction:}}
		\item \red{\textbf{Model Checking over ANSI-C:}}
		\item \red{\textbf{SLAM:}}
	\end{itemize}
\end{itemize}
\part{\red{Partial Order Reduction}}
\part{\red{Timed Automatas}}
\part{\red{Markov Chains}}
\part{\red{K-Induction. Model Checking sobre SAT}}
\part{\red{Model checking LTL propertis over ANSI-C programs with bounded traces}}
\part{\red{SLAM - Abstracciones para realizar model checking sobre C}}
\end{document}



